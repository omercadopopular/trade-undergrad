\documentclass[notes,11pt, aspectratio=169, xcolor=table]{beamer}

\usepackage{pgfpages}
% These slides also contain speaker notes. You can print just the slides,
% just the notes, or both, depending on the setting below. Comment out the want
% you want.
\setbeameroption{hide notes} % Only slide
%\setbeameroption{show only notes} % Only notes
%\setbeameroption{show notes on second screen=right} % Both


\newtheorem{proposition}{Proposition}
\newcommand{\blue}[1]{\textcolor{blue}{#1}}
\newcommand{\white}[1]{\textcolor{white}{#1}}

\usepackage{helvet}
\usepackage[default]{lato}
\usepackage{array}
\usepackage{tikz}
\usetikzlibrary{shapes.geometric}
\usepackage{pgfplots}
\usetikzlibrary{patterns, pgfplots.fillbetween}
\usepackage{graphicx}
\usepackage{verbatim}
\setbeamertemplate{note page}{\pagecolor{yellow!5}\insertnote}
\usetikzlibrary{positioning}
\usetikzlibrary{snakes}
\usetikzlibrary{calc}
\usetikzlibrary{arrows}
\usetikzlibrary{decorations.markings}
\usetikzlibrary{shapes.misc}
\usetikzlibrary{matrix,shapes,arrows,fit,tikzmark}
\usepackage{amsmath}
\usepackage{mathpazo}
\usepackage{hyperref}
\usepackage{lipsum}
\usepackage{multimedia}
\usepackage{graphicx}
\usepackage{multirow}
\usepackage{graphicx}
\usepackage{dcolumn}
\usepackage{bbm}
\usepackage{emoji}
\usepackage[style=authoryear,sorting=nyt,uniquename=false]{biblatex}

\addbibresource{references.bib} 

\newcolumntype{d}[0]{D{.}{.}{5}}

\def\@@mybluebox[#1][#2]#3{
    \sbox\mytempbox{#3}%
    \mytemplen\ht\mytempbox
    \advance\mytemplen #1\relax
    \ht\mytempbox\mytemplen
    \mytemplen\dp\mytempbox
    \advance\mytemplen #2\relax
    \dp\mytempbox\mytemplen
    \colorbox{myblue}{\hspace{1em}\usebox{\mytempbox}\hspace{1em}}}


\usepackage{changepage}
\usepackage{appendixnumberbeamer}
\newcommand{\beginbackup}{
   \newcounter{framenumbervorappendix}
   \setcounter{framenumbervorappendix}{\value{framenumber}}
   \setbeamertemplate{footline}
   {
     \leavevmode%
     \hline
     box{%
       \begin{beamercolorbox}[wd=\paperwidth,ht=2.25ex,dp=1ex,right]{footlinecolor}%
%         \insertframenumber  \hspace*{2ex} 
       \end{beamercolorbox}}%
     \vskip0pt%
   }
 }
\newcommand{\backupend}{
   \addtocounter{framenumbervorappendix}{-\value{framenumber}}
   \addtocounter{framenumber}{\value{framenumbervorappendix}} 
}


\usepackage{graphicx}
\usepackage[space]{grffile}
\usepackage{booktabs}

% These are my colors -- there are many like them, but these ones are mine.
\definecolor{blue}{RGB}{0,114,178}
\definecolor{red}{RGB}{213,94,0}
\definecolor{yellow}{RGB}{240,228,66}
\definecolor{green}{RGB}{0,158,115}

\hypersetup{
  colorlinks=false,
  linkbordercolor = {white},
  linkcolor = {blue}
}


%% I use a beige off white for my background
\definecolor{MyBackground}{RGB}{255,253,218}

%% Uncomment this if you want to change the background color to something else
%\setbeamercolor{background canvas}{bg=MyBackground}

%% Change the bg color to adjust your transition slide background color!
\newenvironment{transitionframe}{
  \setbeamercolor{background canvas}{bg=yellow}
  \begin{frame}}{
    \end{frame}
}

\setbeamercolor{frametitle}{fg=blue}
\setbeamercolor{title}{fg=blue}
\setbeamertemplate{footline}[frame number]
\setbeamertemplate{navigation symbols}{} 
\setbeamertemplate{itemize items}{-}
\setbeamercolor{itemize item}{fg=blue}
\setbeamercolor{itemize subitem}{fg=blue}
\setbeamercolor{enumerate item}{fg=blue}
\setbeamercolor{enumerate subitem}{fg=blue}
\setbeamercolor{button}{bg=MyBackground,fg=blue,}



% If you like road maps, rather than having clutter at the top, have a roadmap show up at the end of each section 
% (and after your introduction)
% Uncomment this is if you want the roadmap!
% \AtBeginSection[]
% {
%    \begin{frame}
%        \frametitle{Roadmap of Talk}
%        \tableofcontents[currentsection]
%    \end{frame}
% }
\setbeamercolor{section in toc}{fg=blue}
\setbeamercolor{subsection in toc}{fg=red}
\setbeamersize{text margin left=1em,text margin right=1em} 

\newenvironment{wideitemize}{\itemize\addtolength{\itemsep}{10pt}}{\enditemize}

\usepackage{environ}
\NewEnviron{videoframe}[1]{
  \begin{frame}
    \vspace{-8pt}
    \begin{columns}[onlytextwidth, T] % align columns
      \begin{column}{.58\textwidth}
        \begin{minipage}[t][\textheight][t]
          {\dimexpr\textwidth}
          \vspace{8pt}
          \hspace{4pt} {\Large \sc \textcolor{blue}{#1}}
          \vspace{8pt}
          
          \BODY
        \end{minipage}
      \end{column}%
      \hfill%
      \begin{column}{.42\textwidth}
        \colorbox{green!20}{\begin{minipage}[t][1.2\textheight][t]
            {\dimexpr\textwidth}
            Face goes here
          \end{minipage}}
      \end{column}%
    \end{columns}
  \end{frame}
}

\title[]{International Trade: Lecture 5}
\subtitle[]{Ricardian Trade with Multiple Goods}
\author[Góes]
{Carlos Góes\inst{1}}
\date{Fall 2025}
\institute[GWU]{\inst{1} George Washington University }



\begin{document}

%%% TIKZ STUFF
\tikzset{   
        every picture/.style={remember picture,baseline},
        every node/.style={anchor=base,align=center,outer sep=1.5pt},
        every path/.style={thick},
        }
\newcommand\marktopleft[1]{%
    \tikz[overlay,remember picture] 
        \node (marker-#1-a) at (-.3em,.3em) {};%
}
\newcommand\markbottomright[2]{%
    \tikz[overlay,remember picture] 
        \node (marker-#1-b) at (0em,0em) {};%
}
\tikzstyle{every picture}+=[remember picture] 
\tikzstyle{mybox} =[draw=black, very thick, rectangle, inner sep=10pt, inner ysep=20pt]
\tikzstyle{fancytitle} =[draw=black,fill=red, text=white]
%%%% END TIKZ STUFF



%----------------------------------------------------------------------%
%-------------------       TITLE PAGE       ---------------------------%
%----------------------------------------------------------------------%





%----------------------------------------------------------------------%






%----------------------------------------------------------------------%
%----------------------------------------------------------------------%

%----------------------------------------------------------------------%
\frame{\titlepage}
\addtocounter{framenumber}{-1}
%----------------------------------------------------------------------%



%----------------------------------------------------------------------%
%----------------------------------------------------------------------%

\begin{frame}{The intellectual history of Ricardian Trade}
\begin{wideitemize}
        \item<1-> Ricardo (1817 On the Principles of Political Economy and Taxation): 2 × 2 trade with two industries and two countries
        \item<2-> Dornbusch, Fischer and Samuelson (1977): 2 x many Ricardian trade either with many industries or with many countries, but not both
        \item<2-> Dornbusch: German-American, Chicago PhD, MIT Prof; Fischer: Israeli-American, MIT PhD, Chief Econ IMF + WB, Fed VP; SamuelsonÇ first American to win Nobel, one of the greatest economists of all time.
        \item<3-> Eaton and Kortum (2002): many x many Ricardian trade with many industries and many countries
        \end{wideitemize}
\end{frame}


\begin{frame}{Ricardian Model with Multiple Goods: Preliminaries}
\begin{wideitemize}
        \item 2 countries $i \in \{ H, F\}$
        \item Many goods indexed (labeled) $g \in [0,1]$
        \item One factor of production: labor $L_i$ mobile between sectors
        \item In country $i$, to produce one unit of good $g$, firms use $a_{i,g}$ units of labor
        \item Key force: differences in local technology (labor productivity)
        \item Trade balances (value of exports = value of imports)
        \end{wideitemize}
\end{frame}

\begin{frame}{Ricardian Model with Multiple Goods: Preliminaries}
\begin{wideitemize}
        \item<1-> What does it mean to have a continuum of goods $g \in [0,1]$?
        \item What is exactly between good $0$ and good $1$? \\
        \qquad \textcolor{gray}{good $1/2$}
        \item<2-> But what is exactly between good $0$ and good $1/2$? \\
        \qquad \textcolor{gray}{good $1/4$}
        
        \item<3-> Do you see where this is going? Each one of the lines below indexes a ``different good'':
        
        \begin{center}
        \begin{tikzpicture}
            % draw horizontal line   
        
            \draw[blue] (0,.95) -- (10,.95);
        
            % draw vertical lines
            \foreach \x in {0,10}
              \draw[blue] (\x cm,{.95+(.205)}) -- (\x cm,{.95-(.205)});
            \foreach \x in {2.5,4.375,5.78125,6.8359375,7.626953125,8.22021484375,8.6651611328125,8.99887084960937,9.24915313720703,9.43686485290527,9.57764863967896,9.68323647975922,9.76242735981941,9.82182051986456,9.86636538989842,9.89977404242381,9.92483053181786,9.9436228988634,9.95771717414755,9.96828788061066,9.976215910458,9.9821619328435,9.98662144963262,9.98996608722447,9.99247456541835,9.99435592406376,9.99576694304782,9.99682520728587,9.9976189054644,9.9982141790983,9.99866063432372}
              \draw (\x cm,{.95+(.205)}) -- (\x cm,{.95-(.205)});
        
            % draw nodes
            \draw (0,0.95) node[below=.105] {\textcolor{blue}{$0$}};
            \draw (10,0.95) node[below=.105] {\textcolor{blue}{$1$}};
          \end{tikzpicture}

    Can find a real number between two different real numbers \\ \qquad $\implies$ infinitely many goods $g \in [0,1]$!

\end{center}
\end{wideitemize}
\end{frame}

\begin{frame}{Perfect Competition, Prices and Unit Costs}
\begin{wideitemize}
        \item<1-> No trade barriers (for now), i.e. domestic prices are equal to prices abroad
        \item In country $i$, firms producing good $g$ maximize profits under perfect competition:
        \begin{equation*}
            \max_{y_{i,g}} \pi_{i,g} = \max_{y_{i,g}} p_{g} y_{i,g} - w_i a_{i,g} y_{i,g} 
        \end{equation*}

        \item<2-> Labor is mobile $\implies$ single wage $w_i$ for workers producing every good in each country 

        \item<3-> Under perfect competition, goods prices reflect unit cost:
        \begin{itemize}
            \item If home makes $g$, its factory-gate price is $P_g = a_{H,g} w_{H}$
            \item If foreign makes $g$, its factory-gate price is $P_g = a_{F,g} w_{F}$
        \end{itemize}
        \end{wideitemize}
\end{frame}

\begin{frame}{Where will production be located?}
\begin{wideitemize}
    \item<1-> In the simple Ricardian model, we conclude that country $F$ has a comparative advantage if $\frac{a_{F,g}}{a_{H,g}} \le  \frac{a_{F,g'}}{a_{H,g'}}$. \\
    \qquad \textcolor{gray}{(what to do with many goods?)}

    \item<2-> \blue{Question}: which country has a comparative advantage in good $g$ if $\frac{w_H}{w_F} \le  \frac{a_{F,g}}{a_{H,g}}$?

    \item<3-> Production only if cheaper to produce domestically than to import goods, i.e.:

    \begin{equation*}
        w_H a_{H,g} \le w_F a_{F,g} \text{ or, equivalently, if } \frac{w_H}{w_F} \le  \frac{a_{F,g}}{a_{H,g}}
    \end{equation*}

    \item<4-> \blue{Prices}: price of good $g$ will be the lowest of unit costs at home or abroad:


        \begin{equation*}
            P_{g} = \min\{w_H a_{H,g}, w_F a_{F,g}\}  \qquad \text{ for } g \in [0,1]
        \end{equation*}
        
        \end{wideitemize}
\end{frame}


\begin{frame}{Where will production be located?}
\begin{wideitemize}
    \item<1-> \blue{Assumption}: order goods $g \in [0,1]$ according to the home country's comparative advantage.

     \item<2-> \blue{Define}: $A_g \equiv  a_{F,g}/ a_{H,g}$. $\implies$ $A_g$ strictly decreasing in $g$.

     \begin{itemize}
         \item $A_0= \frac{a_{F,0}}{a_{H,0}}$: home is the \blue{most productive}
         \item $A_1= \frac{a_{F,1}}{a_{H,1}}$: home is the \blue{least productive} 
     \end{itemize}

            
        \end{wideitemize}
\end{frame}

\begin{frame}{Relative cost schedule}
    \begin{figure}
        \centering
        \begin{tikzpicture}
        \begin{axis}[
            axis lines=left,
            xmin=0, xmax=1.05,
            ymin=0, ymax=1.05,
            ylabel={$A_g$},
            xtick={0,1},
            ytick={0,1},
            xticklabels={0, 1},
            yticklabels={0, 1},
            tick style={draw=none},
            xticklabel style={anchor=north west, yshift=0pt},
            yticklabel style={anchor=south east, xshift=0pt},
            enlargelimits=false,
            clip=false,
            grid=major,
            width=0.7\textwidth,
            height=0.45\textwidth,
            label style={font=\small},
            tick label style={font=\small},
        ]
        

            % Relative cost schedule
            \addplot[red, thick, domain=0.02:1] {x^(-1/6) - 0.975};
           
            \end{axis}
        \end{tikzpicture}
        \caption{Relative unit labor costs $A_g = a_{H,g} / a_{F,g}$}
        \label{fig:Ap}
    \end{figure}
\end{frame}

\begin{frame}{Demand}
\begin{wideitemize}
    \item<1-> \blue{Assumption}: consumers in both countries value each good equally

    \item<2-> Demand for $g$ thus depends on national income $w_iL_i$ and the price of good $g$:

    \begin{equation*}
        q_{i,g} = \frac{w_iL_i}{P_{g}}
    \end{equation*}

    \qquad \textcolor{gray}{(if you want to understand where this comes from, check handout)}

    \item<3-> Note that expenditure on each good does not depend on the specific $g$:

    \begin{equation*}
        P_g q_{i,g} = w_iL_i \qquad \text{ for all goods } g
    \end{equation*}

    \item<4-> Consumers spend the same amount on each good! \textcolor{gray}{(why?)}

    \item<5-> \blue{Question}: how does this relate to Cobb-Douglas preferences we saw before?

    
    \end{wideitemize}
\end{frame}


\begin{frame}{Equilibrium}
\begin{wideitemize}
    \item<1-> In equilibrium there will be some cut-off good $G$ for which:

    \begin{equation*}
        \frac{w_H}{w_F} = A_G
    \end{equation*}

     \item<2-> Since $A_g$ is decreasing in $g$, then:

     \begin{itemize}
         \item Goods $[0,G]$ will be produced at home (for them, $w_H / w_F \le A_g$)
         \item Goods $(G,1]$ will be produced at abroad (for them, $w_H / w_F > A_g$)
        \end{itemize}

        \item<3-> Recall: trade pattern and specialization depends on Home-to-Foreign wage ratio (gap) $w_H/w_F$ 

    \end{wideitemize}
\end{frame}

\begin{frame}{Specialization}
    \begin{figure}
        \centering
        \begin{tikzpicture}
            \begin{axis}[
                axis lines=left,
                xmin=0, xmax=1,
                ymin=0, ymax=1,
                ylabel={\textcolor{red}{$A_g$}, $w_H/w_F$},
                xtick={0,1},
                ytick={0,1},
                xticklabels={0, 1},
                xticklabel style={anchor=north west, yshift=0pt},
                yticklabel style={anchor=south east, xshift=0pt},
                enlargelimits=false,
                clip=false,
                grid=major,
                width=12cm,
                height=8cm,
                every axis plot/.style={thick},
                width=0.7\textwidth,
                height=0.45\textwidth,
                label style={font=\small},
                tick label style={font=\small},
            ]
                       
            \pgfmathsetmacro{\z}{0.05}       
            \pgfmathsetmacro{\c}{0.125}         
            \pgfmathsetmacro{\x}{ (-\z + sqrt(\z^2 + 4*\c*\z)) / (2*\c) }
            \pgfmathsetmacro{\y}{ \z / \x  }         
            \addplot[red, thick, domain=0.05:1] {\z / x}; % A_g
            
            \addplot[dashed] coordinates {(0,\y) (\x,\y)};
            \node at (axis cs:-0.15,\y) {$w_H/w_F=A_G$};
            \addplot[dashed] coordinates {(\x,-.1) (\x,\y)};
            \node at (axis cs:\x,-0.175) {$G$};
            \addplot[gray] coordinates {(0,-0.1) (\x,-0.1)};
            \node at (axis cs:\x/2,-0.075) {produced at $H$};
            \addplot[gray] coordinates {(\x,-0.1) (1,-0.1)};
            \node at (axis cs:( {\x + (1-\x)/2},-0.075) {produced at $F$};
            
            \addplot[only marks, mark=*, color=black, mark size=2pt] coordinates {(\x, \y)};

            \end{axis}
            \end{tikzpicture}
        \caption{Relative unit labor costs $A_g = a_{H,g} / a_{F,g}$}
        \label{fig: Ag}
    \end{figure}

    
\end{frame}

\begin{frame}{Equilibrium determination}
\begin{wideitemize}
    \item<1-> Recall:
    \begin{itemize}
        \item Home produces $G$ share of goods
        \item Foreign produces $1-G$ share of goods
        \item Consumers spend the same amount on each good
    \end{itemize}

    \item<1-> $\implies$ consumers in each country spend $G$ share of their income on goods from $H$

    \item<2-> How to find $G?$ 

    \begin{eqnarray*}
        \underbrace{w_HL_H}_{\text{income}} = \underbrace{G\times w_HL_H + G\times w_FL_F}_{\text{expenditure}} \iff w_HL_H = \frac{G}{1-G}w_FL_F
    \end{eqnarray*}

    \item<3-> Or, solving for $w_H / w_F$:

    \begin{equation*}
        \frac{w_H}{w_F} =  \frac{G}{1-G} \frac{L_F}{L_H}
    \end{equation*}
    
    \end{wideitemize}
\end{frame}


\begin{frame}{Equilibrium determination}
    \begin{figure}
        \centering
        \begin{tikzpicture}
            \begin{axis}[
                axis lines=left,
                xmin=0, xmax=1,
                ymin=0, ymax=1,
                ylabel={\textcolor{red}{$A_g$}, $\frac{g}{1-g} \frac{L_F}{L_H}$},
                xtick={0,1},
                ytick={0,1},
                xticklabels={0, 1},
                xticklabel style={anchor=north west, yshift=0pt},
                yticklabel style={anchor=south east, xshift=0pt},
                enlargelimits=false,
                clip=false,
                grid=major,
                width=12cm,
                height=8cm,
                every axis plot/.style={thick},
                width=0.7\textwidth,
                height=0.45\textwidth,
                label style={font=\small},
                tick label style={font=\small},
            ]
                       
            \pgfmathsetmacro{\z}{0.05}       
            \pgfmathsetmacro{\c}{0.125}         
            \pgfmathsetmacro{\x}{ (-\z + sqrt(\z^2 + 4*\c*\z)) / (2*\c) }
            \pgfmathsetmacro{\y}{ \z / \x  }         
            \addplot[red, thick, domain=0.05:1] {\z / x}; % A_g
            \addplot[black, thick, domain=0.01:0.85] {x / (1 - x) * \c}; % omega(g)
            
            \addplot[dashed] coordinates {(0,\y) (\x,\y)};
            \node at (axis cs:-0.15,\y) {$w_H/w_F=A_G$};
            \addplot[dashed] coordinates {(\x,-.1) (\x,\y)};
            \node at (axis cs:\x,-0.175) {$G$};
            \addplot[gray] coordinates {(0,-0.1) (\x,-0.1)};
            \node at (axis cs:\x/2,-0.075) {produced at $H$};
            \addplot[gray] coordinates {(\x,-0.1) (1,-0.1)};
            \node at (axis cs:( {\x + (1-\x)/2},-0.075) {produced at $F$};
            
            \addplot[only marks, mark=*, color=black, mark size=2pt] coordinates {(\x, \y)};

            \end{axis}
            \end{tikzpicture}
        \caption{Relative unit labor costs $A_g = a_{H,g} / a_{F,g}$}
        \label{fig: Ag}
    \end{figure}
    \end{frame}

 
\begin{frame}{The Great Doubling}
\begin{wideitemize}
    \item<1-> Great Doubling of the global workforce (Freeman 2006):
    \begin{itemize}
        \item in 1980: workforce in open economies roughly 1 billion
        \item in 2000: workforce in open economies roughly 1.5 billion
    \end{itemize}

    \item<2-> Soviet Union collapsed, Russia now in WTO. China shifted to a market economy, now in WTO. India adopted open-market policies.

    \item<3-> How does global specialization change?

    \item<4-> If foreign population increases to $L_F'>L_F$ then:

    \begin{equation*}
        \frac{g}{1-g} \times \frac{L_F'}{L_H} > \frac{g}{1-g} \times \frac{L_F}{L_H} \qquad \text{ for every } g > 0
    \end{equation*}

    
        
    \end{wideitemize}
\end{frame}

\begin{frame}{Increase in Foreign Labor Supply}

    \begin{figure}
        \centering
        \begin{tikzpicture}
            \begin{axis}[
                axis lines=left,
                xmin=0, xmax=1,
                ymin=0, ymax=1,
                ylabel={\textcolor{red}{$A_g$}, \textcolor{gray}{$\frac{g}{1-g} \frac{L_F}{L_H}$}},
                xtick={0,1},
                ytick={0,1},
                xticklabels={0, 1},
                xticklabel style={anchor=north west, yshift=0pt},
                yticklabel style={anchor=south east, xshift=0pt},
                enlargelimits=false,
                clip=false,
                grid=major,
                width=12cm,
                height=8cm,
                every axis plot/.style={thick},
                width=0.7\textwidth,
                height=0.45\textwidth,
                label style={font=\small},
                tick label style={font=\small},
            ]
                       
            \pgfmathsetmacro{\z}{0.05}       
            \pgfmathsetmacro{\c}{0.125}         
            \pgfmathsetmacro{\x}{ (-\z + sqrt(\z^2 + 4*\c*\z)) / (2*\c) }
            \pgfmathsetmacro{\y}{ \z / \x  }         
            \addplot[red, thick, domain=0.05:1] {\z / x}; % A_g
            \addplot[gray, thick, domain=0.01:0.85] {x / (1 - x) * \c}; % omega(g)

            \pgfmathsetmacro{\zn}{0.05}       
            \pgfmathsetmacro{\cn}{0.4}         
            \pgfmathsetmacro{\xn}{ (-\zn + sqrt(\zn^2 + 4*\cn*\zn)) / (2*\cn) }
            \pgfmathsetmacro{\yn}{ \zn / \xn  }         

            \addplot[dashed, gray] coordinates {(0,\y) (\x,\y)};
            \node[gray] at (axis cs:-0.15,\y) {$w_H/w_F=A_G$};
            \addplot[dashed, gray] coordinates {(\x,-.1) (\x,\y)};
            \node[gray] at (axis cs:\x,-0.175) {$G$};
            \addplot[only marks, mark=*, color=gray, mark size=2pt] coordinates {(\x, \y)};



            \addplot[gray] coordinates {(0,-0.1) (\xn,-0.1)};
            \addplot[gray] coordinates {(\xn,-0.1) (1,-0.1)};
            

            \end{axis}
            \end{tikzpicture}
        \caption{Change in relative wages curve}
        \label{fig: Ag-change}
    \end{figure}
\end{frame}


\begin{frame}{Increase in Foreign Labor Supply}
\addtocounter{framenumber}{-1}

    \begin{figure}
        \centering
        \begin{tikzpicture}
            \begin{axis}[
                axis lines=left,
                xmin=0, xmax=1,
                ymin=0, ymax=1,
                ylabel={\textcolor{red}{$A_g$}, \textcolor{gray}{$\frac{g}{1-g} \frac{L_F}{L_H}$}, $\frac{g}{1-g} \frac{L_F'}{L_H}$},
                xtick={0,1},
                ytick={0,1},
                xticklabels={0, 1},
                xticklabel style={anchor=north west, yshift=0pt},
                yticklabel style={anchor=south east, xshift=0pt},
                enlargelimits=false,
                clip=false,
                grid=major,
                width=12cm,
                height=8cm,
                every axis plot/.style={thick},
                width=0.7\textwidth,
                height=0.45\textwidth,
                label style={font=\small},
                tick label style={font=\small},
            ]
                       
            \pgfmathsetmacro{\z}{0.05}       
            \pgfmathsetmacro{\c}{0.125}         
            \pgfmathsetmacro{\x}{ (-\z + sqrt(\z^2 + 4*\c*\z)) / (2*\c) }
            \pgfmathsetmacro{\y}{ \z / \x  }         
            \addplot[red, thick, domain=0.05:1] {\z / x}; % A_g
            \addplot[gray, thick, domain=0.01:0.85] {x / (1 - x) * \c}; % omega(g)

            \pgfmathsetmacro{\zn}{0.05}       
            \pgfmathsetmacro{\cn}{0.4}         
            \pgfmathsetmacro{\xn}{ (-\zn + sqrt(\zn^2 + 4*\cn*\zn)) / (2*\cn) }
            \pgfmathsetmacro{\yn}{ \zn / \xn  }         
            \addplot[black, thick, domain=0.01:0.7] {x / (1 - x) * \cn}; % omega(g)

            \addplot[dashed, gray] coordinates {(0,\y) (\x,\y)};
            \node[gray] at (axis cs:-0.15,\y) {$w_H/w_F=A_G$};
            \addplot[dashed, gray] coordinates {(\x,-.1) (\x,\y)};
            \node[gray] at (axis cs:\x,-0.175) {$G$};
            \addplot[only marks, mark=*, color=gray, mark size=2pt] coordinates {(\x, \y)};



            \addplot[gray] coordinates {(0,-0.1) (\xn,-0.1)};
            \addplot[gray] coordinates {(\xn,-0.1) (1,-0.1)};
            

            \end{axis}
            \end{tikzpicture}
        \caption{Change in relative wages curve}
        \label{fig: Ag-change}
    \end{figure}
\end{frame}

\begin{frame}{Increase in Foreign Labor Supply}
\addtocounter{framenumber}{-1}

    \begin{figure}
        \centering
        \begin{tikzpicture}
            \begin{axis}[
                axis lines=left,
                xmin=0, xmax=1,
                ymin=0, ymax=1,
                ylabel={\textcolor{red}{$A_g$}, \textcolor{gray}{$\frac{g}{1-g} \frac{L_F}{L_H}$}, $\frac{g}{1-g} \frac{L_F'}{L_H}$},
                xtick={0,1},
                ytick={0,1},
                xticklabels={0, 1},
                xticklabel style={anchor=north west, yshift=0pt},
                yticklabel style={anchor=south east, xshift=0pt},
                enlargelimits=false,
                clip=false,
                grid=major,
                width=12cm,
                height=8cm,
                every axis plot/.style={thick},
                width=0.7\textwidth,
                height=0.45\textwidth,
                label style={font=\small},
                tick label style={font=\small},
            ]
                       
            \pgfmathsetmacro{\z}{0.05}       
            \pgfmathsetmacro{\c}{0.125}         
            \pgfmathsetmacro{\x}{ (-\z + sqrt(\z^2 + 4*\c*\z)) / (2*\c) }
            \pgfmathsetmacro{\y}{ \z / \x  }         
            \addplot[red, thick, domain=0.05:1] {\z / x}; % A_g
            \addplot[gray, thick, domain=0.01:0.85] {x / (1 - x) * \c}; % omega(g)

            \pgfmathsetmacro{\zn}{0.05}       
            \pgfmathsetmacro{\cn}{0.4}         
            \pgfmathsetmacro{\xn}{ (-\zn + sqrt(\zn^2 + 4*\cn*\zn)) / (2*\cn) }
            \pgfmathsetmacro{\yn}{ \zn / \xn  }         
            \addplot[black, thick, domain=0.01:0.7] {x / (1 - x) * \cn}; % omega(g)

            \addplot[dashed, gray] coordinates {(0,\y) (\x,\y)};
            \node[gray] at (axis cs:-0.15,\y) {$w_H/w_F=A_G$};
            \addplot[dashed, gray] coordinates {(\x,-.1) (\x,\y)};
            \node[gray] at (axis cs:\x,-0.175) {$G$};
            \addplot[only marks, mark=*, color=gray, mark size=2pt] coordinates {(\x, \y)};


            \addplot[dashed] coordinates {(0,\yn) (\xn,\yn)};
            \node at (axis cs:-0.175,\yn) {$(w_H/w_F)'=A'_G$};
            \addplot[dashed] coordinates {(\xn,-.1) (\xn,\yn)};
            \node at (axis cs:\xn,-0.175) {$G'$};
            \addplot[gray] coordinates {(0,-0.1) (\xn,-0.1)};
            \addplot[gray] coordinates {(\xn,-0.1) (1,-0.1)};
            \addplot[only marks, mark=*, color=black, mark size=2pt] coordinates {(\xn, \yn)};
            

            \end{axis}
            \end{tikzpicture}
        \caption{Change in relative wages curve}
        \label{fig: Ag-change}
    \end{figure}

\end{frame}
   


\begin{frame}{Home gains from population growth}
\begin{wideitemize}
    \item<1-> Foreign labor supply causes industry shutdowns at Home... \\
    \qquad ...but raises the relative Home wage $(w_H/w_F)'>(w_H/w_F)$

    \item<2-> Does welfare go up? Need to check real income $w_H / p_g$ for each product $g \in [0,1]$
    \begin{wideitemize}
        \item<3-> Goods $g \in [0,G']$, sourced domestically, no change: \\ $(w_H / p_g)'= w'_H / (w_H' a_{H,g}) = 1/a_{H,g} = w_H / (w_H a_{H,g}) = w_H / p_g$ 
        \item<4-> Goods $g \in [G,1]$, sourced abroad, gains: \\ $(w_H / p_g)'= w'_H / (w_F' a_{F,g}) > w_H / (w_F a_{F,g}) = w_H / p_g$
        \item<5-> Goods $g \in [G',G]$, pivotal (change sources), gains:
        \begin{eqnarray*}
            (w_H / p_g)' &\ge& (w_H / p_g) \\
            \iff w_H' / (w_F' a_{F,g}) &\ge& w_H / (w_H a_{H,g}) \\
            \iff w_H' / w_F'  &\ge& a_{F,g} / a_{H,g} = A_g
        \end{eqnarray*}
        \end{wideitemize}
    
    \end{wideitemize}
\end{frame}

\begin{frame}{Welfare effects}

    \begin{figure}
        \centering
        \begin{tikzpicture}
            \begin{axis}[
                axis lines=left,
                xmin=0, xmax=1,
                ymin=0, ymax=1,
                ylabel={\textcolor{red}{$A_g$}, \textcolor{gray}{$\frac{g}{1-g} \frac{L_F}{L_H}$}, $\frac{g}{1-g} \frac{L_F'}{L_H}$},
                xtick={0,1},
                ytick={0,1},
                xticklabels={0, 1},
                xticklabel style={anchor=north west, yshift=0pt},
                yticklabel style={anchor=south east, xshift=0pt},
                enlargelimits=false,
                clip=false,
                grid=major,
                width=12cm,
                height=8cm,
                every axis plot/.style={thick},
                width=0.7\textwidth,
                height=0.45\textwidth,
                label style={font=\small},
                tick label style={font=\small},
            ]
                       
            \pgfmathsetmacro{\z}{0.05}       
            \pgfmathsetmacro{\c}{0.125}         
            \pgfmathsetmacro{\x}{ (-\z + sqrt(\z^2 + 4*\c*\z)) / (2*\c) }
            \pgfmathsetmacro{\y}{ \z / \x  }         
            \addplot[red, thick, domain=0.05:1] {\z / x}; % A_g
            \addplot[gray, thick, domain=0.01:0.85] {x / (1 - x) * \c}; % omega(g)

            \pgfmathsetmacro{\zn}{0.05}       
            \pgfmathsetmacro{\cn}{0.4}         
            \pgfmathsetmacro{\xn}{ (-\zn + sqrt(\zn^2 + 4*\cn*\zn)) / (2*\cn) }
            \pgfmathsetmacro{\yn}{ \zn / \xn  }         
            \addplot[black, thick, domain=0.01:0.7] {x / (1 - x) * \cn}; % omega(g)

            \addplot[dashed, gray] coordinates {(0,\y) (\x,\y)};
            \addplot[dashed] coordinates {(\x,\y) (1,\y)};
            \node[gray] at (axis cs:\x,-0.175) {$G$};
            \node[gray] at (axis cs:-0.125,\y) {$(w_H/w_F)$};
            \addplot[dashed, gray] coordinates {(\x,-.1) (\x,\yn)};
            \addplot[only marks, mark=*, color=gray, mark size=2pt] coordinates {(\x, \y)};


            \addplot[dashed] coordinates {(0,\yn) (1,\yn)};
            \node at (axis cs:-0.1,\yn) {$(w_H/w_F)'$};
            \addplot[dashed, gray] coordinates {(\xn,-.1) (\xn,\yn)};
            \node at (axis cs:\xn,-0.175) {$G'$};
            \addplot[gray] coordinates {(0,-0.1) (\xn,-0.1)};
            \addplot[gray] coordinates {(\xn,-0.1) (1,-0.1)};
            \addplot[only marks, mark=*, color=black, mark size=2pt] coordinates {(\xn, \yn)};

            %fill area            
            \path[name path=C] (axis cs:\x,\yn) -- (axis cs:1,\yn);
            \path[name path=D] (axis cs:\x,\y) -- (axis cs:1,\y);

            \addplot [
                fill=red!20,
                draw=none
            ] fill between [of=C and D];

            \end{axis}
            \end{tikzpicture}
        \caption{Change in relative wages curve}
        \label{fig: Ag-change}
    \end{figure}

\end{frame}

   
\begin{frame}{Welfare effects}
\addtocounter{framenumber}{-1}

    \begin{figure}
        \centering
        \begin{tikzpicture}
            \begin{axis}[
                axis lines=left,
                xmin=0, xmax=1,
                ymin=0, ymax=1,
                ylabel={\textcolor{red}{$A_g$}, \textcolor{gray}{$\frac{g}{1-g} \frac{L_F}{L_H}$}, $\frac{g}{1-g} \frac{L_F'}{L_H}$},
                xtick={0,1},
                ytick={0,1},
                xticklabels={0, 1},
                xticklabel style={anchor=north west, yshift=0pt},
                yticklabel style={anchor=south east, xshift=0pt},
                enlargelimits=false,
                clip=false,
                grid=major,
                width=12cm,
                height=8cm,
                every axis plot/.style={thick},
                width=0.7\textwidth,
                height=0.45\textwidth,
                label style={font=\small},
                tick label style={font=\small},
            ]
                       
            \pgfmathsetmacro{\z}{0.05}       
            \pgfmathsetmacro{\c}{0.125}         
            \pgfmathsetmacro{\x}{ (-\z + sqrt(\z^2 + 4*\c*\z)) / (2*\c) }
            \pgfmathsetmacro{\y}{ \z / \x  }         
            \addplot[red, thick, domain=0.05:1] {\z / x}; % A_g
            \addplot[gray, thick, domain=0.01:0.85] {x / (1 - x) * \c}; % omega(g)

            \pgfmathsetmacro{\zn}{0.05}       
            \pgfmathsetmacro{\cn}{0.4}         
            \pgfmathsetmacro{\xn}{ (-\zn + sqrt(\zn^2 + 4*\cn*\zn)) / (2*\cn) }
            \pgfmathsetmacro{\yn}{ \zn / \xn  }         
            \addplot[black, thick, domain=0.01:0.7] {x / (1 - x) * \cn}; % omega(g)

            \addplot[dashed, gray] coordinates {(0,\y) (\x,\y)};
            \addplot[dashed] coordinates {(\x,\y) (1,\y)};
            \node[gray] at (axis cs:\x,-0.175) {$G$};
            \node[gray] at (axis cs:-0.125,\y) {$(w_H/w_F)$};
            \addplot[dashed, gray] coordinates {(\x,-.1) (\x,\yn)};
            \addplot[only marks, mark=*, color=gray, mark size=2pt] coordinates {(\x, \y)};


            \addplot[dashed] coordinates {(0,\yn) (1,\yn)};
            \node at (axis cs:-0.1,\yn) {$(w_H/w_F)'$};
            \addplot[dashed, gray] coordinates {(\xn,-.1) (\xn,\yn)};
            \node at (axis cs:\xn,-0.175) {$G'$};
            \addplot[gray] coordinates {(0,-0.1) (\xn,-0.1)};
            \addplot[gray] coordinates {(\xn,-0.1) (1,-0.1)};
            \addplot[only marks, mark=*, color=black, mark size=2pt] coordinates {(\xn, \yn)};

            %fill area            
            \path[name path=A] (axis cs:\x,\yn) -- (axis cs:\xn,\yn);
            \addplot [
                name path=B,
                domain=\x:\xn,
                draw=none
            ] {\z / x};
            
            \addplot [
                fill=cyan!20,
                draw=none
            ] fill between [of=A and B];
            
            \path[name path=C] (axis cs:\x,\yn) -- (axis cs:1,\yn);
            \path[name path=D] (axis cs:\x,\y) -- (axis cs:1,\y);

            \addplot [
                fill=red!20,
                draw=none
            ] fill between [of=C and D];

            \end{axis}
            \end{tikzpicture}
        \caption{Change in relative wages curve}
        \label{fig: Ag-change}
    \end{figure}

\end{frame}
   

\begin{frame}{Trade Barriers and Prices}
\begin{wideitemize}
    \item<1-> Relax the assumption of costless trade...
    \begin{itemize}
        \item sellers have to pay for shipping and insurance
        \item goods may get lost or perish along the way (e.g., avocados being shipped internationally)
    \end{itemize}
    \item<2-> \blue{Iceberg trade costs}: for each unit of a good to arrive at a destination, $\tau>1$ units need to be shipped from the origin country. \\
    \qquad \textcolor{gray}{(as if goods melt away between origin and destination)}

    \item<3-> Production at home only if cheaper to produce domestically than import, i.e.: 

    \begin{equation*}
        w_H a_{H,g} \le \tau w_F a_{F,g} \text{ or, equivalently, if } \frac{w_H}{w_F} \le  \tau  \frac{a_{F,g}}{a_{H,g}} = \tau A_g
    \end{equation*}

    \item<4-> Not every good produced at home will be exported; only if:

\begin{equation*}
        \tau w_H a_{H,g} \le w_F a_{F,g} \text{ or, equivalently, if } \frac{w_H}{w_F} \le  \frac{1}{\tau} \frac{a_{F,g}}{a_{H,g}} \equiv \frac{1}{\tau} A_g
\end{equation*}

    

    \end{wideitemize}
\end{frame}


\begin{frame}{Specialization with Trade Barriers}
\begin{wideitemize}
    \item<1-> Now, cut-offs:

    \begin{itemize}
        \item Foreign will export all goods satisfying $\frac{w_H}{w_F} >  \tau A_{G_H}$, comprising  $[G_H,1]$
        \item Home will export all goods satisfying $\frac{w_H}{w_F} \le  A_{G_F} / \tau$, comprising   $[0,G_F]$
        \item Goods $(G_F, G_H)$ will be produced and consumed in each country but not exported
    \end{itemize}
    
    
    \item<2-> Again, these cut-offs are implicitly defined through a function:


\begin{equation*}
    \frac{w_H}{w_F} =  \frac{G_H}{1-G_F} \times \frac{ L_F }{L_H}
\end{equation*}

    \end{wideitemize}
\end{frame}


\begin{frame}{Equilibrium with Trade Costs}
    \begin{figure}
        \centering
        \begin{tikzpicture}
            \begin{axis}[
                axis lines=left,
                xmin=0, xmax=1,
                ymin=0, ymax=1,
                ylabel={\textcolor{red}{$\tau A_g$}, \textcolor{blue}{$A_g / \tau$}},
                xtick={0,1},
                ytick={0,1},
                xticklabels={0, 1},
                xticklabel style={anchor=north west, yshift=0pt},
                yticklabel style={anchor=south east, xshift=0pt},
                enlargelimits=false,
                clip=false,
                grid=major,
                width=12cm,
                height=8cm,
                every axis plot/.style={thick},
                width=0.7\textwidth,
                height=0.45\textwidth,
                label style={font=\small},
                tick label style={font=\small},
            ]
                       
            \pgfmathsetmacro{\t}{1.25}       
            \pgfmathsetmacro{\z}{0.075}       
            \pgfmathsetmacro{\c}{0.125}         
            \pgfmathsetmacro{\x}{(-(\z/\t) + sqrt((\z/\t)^2 + 4*\c*\z/\t)) / (2*\c)}
            \pgfmathsetmacro{\y}{(\z / \x) / \t}
            \pgfmathsetmacro{\xf}{\z*\t/\y}
            \addplot[red, thick, domain=0.15:1] {( \z / x ) * \t}; %home
            \addplot[blue, thick, domain=0.15:1] {\z / (x * \t )}; %foreign
            \addplot[dashed, gray, thick, domain=0.15:1] {\z / (x)}; %foreign
            


            \end{axis}
            \end{tikzpicture}
        \caption{Trade equilibrium with positive trade costs $\tau$}
        \label{fig: Ag-tariffs}
    \end{figure}

    
\end{frame}


\begin{frame}{Equilibrium with Trade Costs}
    \begin{figure}
        \centering
        \begin{tikzpicture}
            \begin{axis}[
                axis lines=left,
                xmin=0, xmax=1,
                ymin=0, ymax=1,
                ylabel={\textcolor{red}{$\tau A_g$}, \textcolor{blue}{$A_g / \tau$}, $w_H/w_F$},
                xtick={0,1},
                ytick={0,1},
                xticklabels={0, 1},
                xticklabel style={anchor=north west, yshift=0pt},
                yticklabel style={anchor=south east, xshift=0pt},
                enlargelimits=false,
                clip=false,
                grid=major,
                width=12cm,
                height=8cm,
                every axis plot/.style={thick},
                width=0.7\textwidth,
                height=0.45\textwidth,
                label style={font=\small},
                tick label style={font=\small},
            ]
                       
            \pgfmathsetmacro{\t}{1.25}       
            \pgfmathsetmacro{\z}{0.075}       
            \pgfmathsetmacro{\c}{0.125}         
            \pgfmathsetmacro{\x}{(-(\z/\t) + sqrt((\z/\t)^2 + 4*\c*\z/\t)) / (2*\c)}
            \pgfmathsetmacro{\y}{(\z / \x) / \t}
            \pgfmathsetmacro{\xf}{\z*\t/\y}
            \addplot[red, thick, domain=0.15:1] {( \z / x ) * \t}; %home
            \addplot[blue, thick, domain=0.15:1] {\z / (x * \t )}; %foreign
            \addplot[black, thick, domain=0.01:0.85] {x / (1 - x) * \c}; % omega(g)
            


            \end{axis}
            \end{tikzpicture}
        \caption{Trade equilibrium with positive trade costs $\tau$}
        \label{fig: Ag-tariffs}
    \end{figure}

    
\end{frame}


\begin{frame}{Equilibrium with Trade Costs}
    \begin{figure}
        \centering
        \begin{tikzpicture}
            \begin{axis}[
                axis lines=left,
                xmin=0, xmax=1,
                ymin=0, ymax=1,
                ylabel={\textcolor{red}{$\tau A_g$}, \textcolor{blue}{$A_g / \tau$}, $w_H/w_F$},
                xtick={0,1},
                ytick={0,1},
                xticklabels={0, 1},
                xticklabel style={anchor=north west, yshift=0pt},
                yticklabel style={anchor=south east, xshift=0pt},
                enlargelimits=false,
                clip=false,
                grid=major,
                width=12cm,
                height=8cm,
                every axis plot/.style={thick},
                width=0.7\textwidth,
                height=0.45\textwidth,
                label style={font=\small},
                tick label style={font=\small},
            ]
                       
            \pgfmathsetmacro{\t}{1.25}       
            \pgfmathsetmacro{\z}{0.075}       
            \pgfmathsetmacro{\c}{0.125}         
            \pgfmathsetmacro{\x}{(-(\z/\t) + sqrt((\z/\t)^2 + 4*\c*\z/\t)) / (2*\c)}
            \pgfmathsetmacro{\y}{(\z / \x) / \t}
            \pgfmathsetmacro{\xf}{\z*\t/\y}
            \addplot[red, thick, domain=0.15:1] {( \z / x ) * \t}; %home
            \addplot[blue, thick, domain=0.15:1] {\z / (x * \t )}; %foreign
            \addplot[black, thick, domain=0.01:0.85] {x / (1 - x) * \c}; % omega(g)
            
            \addplot[gray] coordinates {(0,-0.1) (1,-0.1)};
            \addplot[dashed] coordinates {(0,\y) (\xf,\y)};
            \node at (axis cs:-0.15,\y) {$w_H/w_F=A_G$};
            \addplot[dashed] coordinates {(\x,-.1) (\x,\y)};
            \node at (axis cs:\x,-0.175) {$G_F$};
            \node at (axis cs:\x/2,-0.075) {$H$ exports};
            \addplot[dashed] coordinates {(\xf,-.1) (\xf,\y)};
            \node at (axis cs:( {\x + (\xf-\x)/2},-0.075) {not traded};
            \node at (axis cs:\xf,-0.175) {$G_H$};
            \node at (axis cs:( {\xf + (1-\xf)/2},-0.075) {$F$ exports};
            
            \addplot[only marks, mark=*, color=black, mark size=2pt] coordinates {(\x, \y)};
            \addplot[only marks, mark=*, color=black, mark size=2pt] coordinates {(\xf, \y)};

            \end{axis}
            \end{tikzpicture}
        \caption{Trade equilibrium with positive trade costs $\tau$}
        \label{fig: Ag-tariffs}
    \end{figure}

    
\end{frame}

\end{document}
