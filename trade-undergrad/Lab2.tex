\documentclass[notes,11pt, aspectratio=169, xcolor=table]{beamer}

% --- [rest of your original preamble] ---

\title[]{International Trade: Data Lab 2}
\subtitle[]{Trade with Data}
\author[Góes]
{Carlos Góes\inst{1}}
\date{Fall 2025}
\institute[GWU]{\inst{1} George Washington University }

\begin{document}

% --- Lab Objective Slide ---
\begin{frame}{Lab Objective & Roadmap}
\textbf{Goal:} Analyze the impact of trade openness on GDP per capita growth using real-world data.
\begin{itemize}
    \item Load and clean global trade and growth data.
    \item Transform and merge datasets for analysis.
    \item Visualize and interpret results using Python.
    \item Reflect on analysis and results.
\end{itemize}
\end{frame}

% --- Step 1: Import Libraries ---
\begin{frame}[fragile]{Step 1: Import Essential Libraries}
\begin{minted}[fontsize=\footnotesize]{python}
# Import dependencies
import pandas as pd
import matplotlib.pyplot as plt
\end{minted}
\note{
\begin{itemize}
\item Why do we use pandas for data analysis? What is matplotlib typically used for?
\item Example: Try importing a library not installed and see the error.
\end{itemize}
}
\end{frame}

% --- Step 2: Declare Data Paths ---
\begin{frame}[fragile]{Step 2: Declare Data Paths}
\begin{minted}[fontsize=\footnotesize]{python}
# Declare paths
tradePath = 'https://github.com/omercadopopular/cgoes/raw/refs/heads/master/trade-undergrad/data/API_NE.TRD.GNFS.ZS_DS2_en_excel_v2_38352.xls'
growthPath = 'https://github.com/omercadopopular/cgoes/raw/refs/heads/master/trade-undergrad/data/API_NY.GDP.PCAP.KD.ZG_DS2_en_excel_v2_122434.xls'
\end{minted}
\note{
\begin{itemize}
\item What would change if you set skiprows=0 when loading?
\item Try loading another Excel file: what errors do you encounter?
\end{itemize}
}
\end{frame}

% --- Step 3: Load and Preview Data ---
\begin{frame}[fragile]{Step 3: Load and Preview Data}
\begin{minted}[fontsize=\footnotesize]{python}
# read the trade and growth data from Excel files
trade = pd.read_excel(tradePath, skiprows=3).drop(columns=['Indicator Name', 'Indicator Code'])
growth = pd.read_excel(growthPath, skiprows=3).drop(columns=['Indicator Name', 'Indicator Code'])
trade
\end{minted}
\vspace{0.5ex}
\textbf{Expected output (from notebook):}
\begin{verbatim}
Country Name Country Code       1960       1961  ...
...
[266 rows x 67 columns]
\end{verbatim}
\note{
\begin{itemize}
\item Identify a row with missing values. What does NaN mean?
\item How many countries are present in the trade dataset?
\end{itemize}
}
\end{frame}

% --- Step 4: Reshape Data to Long Format ---
\begin{frame}[fragile]{Step 4: Reshape Data to Long Format}
\begin{minted}[fontsize=\footnotesize]{python}
# collapse from wide to long format for trade data
trade = trade.melt(id_vars=['Country Name', 'Country Code'],
                   var_name='Year', value_name='Trade')
growth = growth.melt(id_vars=['Country Name', 'Country Code'],
                   var_name='Year', value_name='Growth')
trade
\end{minted}
\vspace{0.5ex}
\textbf{Expected output (from notebook):}
\begin{verbatim}
Country Name Country Code  Year       Trade
...
[17290 rows x 4 columns]
\end{verbatim}
\note{
\begin{itemize}
\item What is the difference between "wide" and "long" data formats?
\item Why do we need to reshape the data before merging?
\end{itemize}
}
\end{frame}

% --- Step 5: Merge DataFrames ---
\begin{frame}[fragile]{Step 5: Merge Trade and Growth Data}
\begin{minted}[fontsize=\footnotesize]{python}
# merge the two dataframes on Country Name, Country Code, and Year
frame = pd.merge(trade, growth, on=['Country Name', 'Country Code', 'Year'])
frame
\end{minted}
\vspace{0.5ex}
\textbf{Expected output (from notebook):}
\begin{verbatim}
Country Name Country Code  Year       Trade     Growth
...
[17290 rows x 5 columns]
\end{verbatim}
\note{
\begin{itemize}
\item What does an "inner join" mean in pandas?
\item Find a country and year where either Trade or Growth is missing.
\end{itemize}
}
\end{frame}

% --- Step 6: Remove Missing Data ---
\begin{frame}[fragile]{Step 6: Remove Rows with Missing Values}
\begin{minted}[fontsize=\footnotesize]{python}
# drop missing values
frame = frame.dropna(subset=['Trade','Growth'])
frame
\end{minted}
\vspace{0.5ex}
\textbf{Expected output (from notebook):}
\begin{verbatim}
Country Name Country Code  Year       Trade     Growth
...
[10871 rows x 5 columns]
\end{verbatim}
\note{
\begin{itemize}
\item What happens if you drop rows with missing data in only one column?
\item How many rows remain after this operation?
\end{itemize}
}
\end{frame}

% --- Step 7: Median Split ---
\begin{frame}[fragile]{Step 7: Categorize Countries by Trade Openness}
\begin{minted}[fontsize=\footnotesize]{python}
# group by year and compute median trade share
median_trade = (frame.groupby(['Year'])['Trade'].transform('median'))

# create a group label (Above / Below median)
frame['group'] = (frame['Trade'] > median_trade).map({True: 'Above Median', False: 'Below Median'})
frame
\end{minted}
\vspace{0.5ex}
\textbf{Expected output (from notebook):}
\begin{verbatim}
Country Name Country Code  Year       Trade     Growth         group
...
[10871 rows x 6 columns]
\end{verbatim}
\note{
\begin{itemize}
\item For a given year, which country has the highest trade share?
\item What are the implications of splitting the dataset at the median?
\end{itemize}
}
\end{frame}

% --- Step 8: Compute Growth Averages ---
\begin{frame}[fragile]{Step 8: Calculate Average Growth by Group}
\begin{minted}[fontsize=\footnotesize]{python}
# take averages across groups
result = frame.groupby(['Year', 'group'])['Growth'].mean().reset_index()
result
\end{minted}
\vspace{0.5ex}
\textbf{Expected output (from notebook):}
\begin{verbatim}
Year         group    Growth
...
[128 rows x 3 columns]
\end{verbatim}
\note{
\begin{itemize}
\item Does "Above Median" group always have higher average growth?
\item Plot one year's growth rates for both groups.
\end{itemize}
}
\end{frame}

% --- Step 9: Pivot Table ---
\begin{frame}[fragile]{Step 9: Create Pivot Table for Visualization}
\begin{minted}[fontsize=\footnotesize]{python}
# pivot table (as in excel)
pivot = result.pivot(index='Year',
                     columns='group',
                     values='Growth')
pivot.index = pivot.index.astype(int)
pivot['difference'] = pivot['Above Median'] - pivot['Below Median']
pivot
\end{minted}
\vspace{0.5ex}
\textbf{Expected output (from notebook):}
\begin{verbatim}
group  Above Median  Below Median  difference
Year
1961   ...          ...           ...
...
2024   ...          ...           ...
\end{verbatim}
\note{
\begin{itemize}
\item For the year 2020, what is the difference in growth between groups?
\item What does a positive difference mean?
\end{itemize}
}
\end{frame}

% --- Step 10: Visualize Results ---
\begin{frame}[fragile]{Step 10: Visualize Growth Differences}
\begin{minted}[fontsize=\footnotesize]{python}
# chart difference
plt.figure(figsize=(10, 6))
plt.plot(pivot.index, pivot['difference'], marker='o')
plt.axhline(pivot['difference'].mean(), color='red', linewidth=1)  # mean line
plt.axhline(0, color='black', linewidth=1)  # zero line
plt.title('Difference in GDP per Capita Growth by High and Low Trade Country', fontsize=14)
plt.xlabel('Year')
plt.ylabel('GDP per Capita Growth (%)')
plt.xticks(rotation=90)
plt.grid(True, linestyle='--', alpha=0.7)
plt.tight_layout()
plt.show()
\end{minted}
\vspace{0.5ex}
\textbf{Expected output (from notebook):}
\begin{verbatim}
<Figure size 1000x600 with 1 Axes>
\end{verbatim}
\note{
\begin{itemize}
\item What does the red line represent?
\item In which years is the difference most pronounced?
\end{itemize}
}
\end{frame}

% --- Final Reflection Slide ---
\begin{frame}{Reflect and Discuss}
\textbf{Questions:}
\begin{itemize}
    \item What does the analysis suggest about the relationship between trade openness and growth?
    \item What other factors might influence these results?
    \item How would you improve this analysis?
\end{itemize}
\end{frame}

\end{document}