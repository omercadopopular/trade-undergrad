\documentclass[notes,11pt, aspectratio=169, xcolor=table]{beamer}

\usepackage{pgfpages}
% These slides also contain speaker notes. You can print just the slides,
% just the notes, or both, depending on the setting below. Comment out the want
% you want.
\setbeameroption{hide notes} % Only slide
%\setbeameroption{show only notes} % Only notes
%\setbeameroption{show notes on second screen=right} % Both

\usepackage{helvet}
\usepackage[default]{lato}
\usepackage{array}
\usepackage{minted}

\newtheorem{proposition}{Proposition}

\usepackage{tikz}
\usetikzlibrary{shapes.geometric}
\usepackage{pgfplots}
\usepackage{graphicx}
\usepackage{verbatim}
\setbeamertemplate{note page}{\pagecolor{yellow!5}\insertnote}
\usetikzlibrary{positioning}
\usetikzlibrary{snakes}
\usetikzlibrary{calc}
\usetikzlibrary{arrows}
\usetikzlibrary{decorations.markings}
\usetikzlibrary{shapes.misc}
\usetikzlibrary{matrix,shapes,arrows,fit,tikzmark}
\usepackage{amsmath}
\usepackage{mathpazo}
\usepackage{hyperref}
\usepackage{lipsum}
\usepackage{multimedia}
\usepackage{graphicx}
\usepackage{multirow}
\usepackage{graphicx}
\usepackage{dcolumn}
\usepackage{bbm}
\usepackage[style=authoryear,sorting=nyt,uniquename=false]{biblatex}

\addbibresource{references.bib} 

\newcolumntype{d}[0]{D{.}{.}{5}}

\def\@@mybluebox[#1][#2]#3{
    \sbox\mytempbox{#3}%
    \mytemplen\ht\mytempbox
    \advance\mytemplen #1\relax
    \ht\mytempbox\mytemplen
    \mytemplen\dp\mytempbox
    \advance\mytemplen #2\relax
    \dp\mytempbox\mytemplen
    \colorbox{myblue}{\hspace{1em}\usebox{\mytempbox}\hspace{1em}}}


\usepackage{changepage}
\usepackage{appendixnumberbeamer}
\newcommand{\beginbackup}{
   \newcounter{framenumbervorappendix}
   \setcounter{framenumbervorappendix}{\value{framenumber}}
   \setbeamertemplate{footline}
   {
     \leavevmode%
     \hline
     box{%
       \begin{beamercolorbox}[wd=\paperwidth,ht=2.25ex,dp=1ex,right]{footlinecolor}%
%         \insertframenumber  \hspace*{2ex} 
       \end{beamercolorbox}}%
     \vskip0pt%
   }
 }
\newcommand{\backupend}{
   \addtocounter{framenumbervorappendix}{-\value{framenumber}}
   \addtocounter{framenumber}{\value{framenumbervorappendix}} 
}


\usepackage{graphicx}
\usepackage[space]{grffile}
\usepackage{booktabs}

% These are my colors -- there are many like them, but these ones are mine.
\definecolor{blue}{RGB}{0,114,178}
\definecolor{red}{RGB}{213,94,0}
\definecolor{yellow}{RGB}{240,228,66}
\definecolor{green}{RGB}{0,158,115}

\hypersetup{
  colorlinks=false,
  linkbordercolor = {white},
  linkcolor = {blue}
}


%% I use a beige off white for my background
\definecolor{MyBackground}{RGB}{255,253,218}

%% Uncomment this if you want to change the background color to something else
%\setbeamercolor{background canvas}{bg=MyBackground}

%% Change the bg color to adjust your transition slide background color!
\newenvironment{transitionframe}{
  \setbeamercolor{background canvas}{bg=yellow}
  \begin{frame}}{
    \end{frame}
}

\setbeamercolor{frametitle}{fg=blue}
\setbeamercolor{title}{fg=blue}
\setbeamertemplate{footline}[frame number]
\setbeamertemplate{navigation symbols}{} 
\setbeamertemplate{itemize items}{-}
\setbeamercolor{itemize item}{fg=blue}
\setbeamercolor{itemize subitem}{fg=blue}
\setbeamercolor{enumerate item}{fg=blue}
\setbeamercolor{enumerate subitem}{fg=blue}
\setbeamercolor{button}{bg=MyBackground,fg=blue,}



% If you like road maps, rather than having clutter at the top, have a roadmap show up at the end of each section 
% (and after your introduction)
% Uncomment this is if you want the roadmap!
% \AtBeginSection[]
% {
%    \begin{frame}
%        \frametitle{Roadmap of Talk}
%        \tableofcontents[currentsection]
%    \end{frame}
% }
\setbeamercolor{section in toc}{fg=blue}
\setbeamercolor{subsection in toc}{fg=red}
\setbeamersize{text margin left=1em,text margin right=1em} 

\newenvironment{wideitemize}{\itemize\addtolength{\itemsep}{10pt}}{\enditemize}

\usepackage{environ}
\NewEnviron{videoframe}[1]{
  \begin{frame}
    \vspace{-8pt}
    \begin{columns}[onlytextwidth, T] % align columns
      \begin{column}{.58\textwidth}
        \begin{minipage}[t][\textheight][t]
          {\dimexpr\textwidth}
          \vspace{8pt}
          \hspace{4pt} {\Large \sc \textcolor{blue}{#1}}
          \vspace{8pt}
          
          \BODY
        \end{minipage}
      \end{column}%
      \hfill%
      \begin{column}{.42\textwidth}
        \colorbox{green!20}{\begin{minipage}[t][1.2\textheight][t]
            {\dimexpr\textwidth}
            Face goes here
          \end{minipage}}
      \end{column}%
    \end{columns}
  \end{frame}
}

\title[]{International Trade: Data Lab 1}
\subtitle[]{Intro to Python}
\author[Góes]
{Carlos Góes\inst{1}}
\date{Fall 2025}
\institute[GWU]{\inst{1} George Washington University }



\begin{document}

%%% TIKZ STUFF
\tikzset{   
        every picture/.style={remember picture,baseline},
        every node/.style={anchor=base,align=center,outer sep=1.5pt},
        every path/.style={thick},
        }
\newcommand\marktopleft[1]{%
    \tikz[overlay,remember picture] 
        \node (marker-#1-a) at (-.3em,.3em) {};%
}
\newcommand\markbottomright[2]{%
    \tikz[overlay,remember picture] 
        \node (marker-#1-b) at (0em,0em) {};%
}
\tikzstyle{every picture}+=[remember picture] 
\tikzstyle{mybox} =[draw=black, very thick, rectangle, inner sep=10pt, inner ysep=20pt]
\tikzstyle{fancytitle} =[draw=black,fill=red, text=white]
%%%% END TIKZ STUFF



%----------------------------------------------------------------------%
%-------------------       TITLE PAGE       ---------------------------%
%----------------------------------------------------------------------%





%----------------------------------------------------------------------%






%----------------------------------------------------------------------%
%----------------------------------------------------------------------%

%----------------------------------------------------------------------%
%----------------------------------------------------------------------%



%----------------------------------------------------------------------%
%----------------------------------------------------------------------%

% --- [rest of your original preamble] ---
\begin{document}

\title[]{International Trade: Data Lab 2}
\subtitle[]{Intro to \mintinline{python}{pandas}; Trade with data}
\author[Góes]
{Carlos Góes\inst{1}}
\date{Fall 2025}
\institute[GWU]{\inst{1} George Washington University }


\frame{\titlepage}
\addtocounter{framenumber}{-1}


% --- Lab Objective Slide ---
\begin{frame}{Roadmap}
\textbf{Question:} How is trade openness related to GDP per capita growth? using real-world data.
\textbf{Method:} Using real-world data to explore this relationship.
\begin{itemize}
    \item Load and clean global trade and growth data.
    \item Transform and merge datasets for analysis.
    \item Visualize and interpret results using Python.
    \item Reflect on analysis and results.
\end{itemize}
\end{frame}

% --- Step 1: Import Libraries ---
\begin{frame}[fragile]{Step 1: Import Essential Libraries}
\begin{minted}[fontsize=\scriptsize]{python}
# Import dependencies
import pandas as pd
import matplotlib.pyplot as plt
\end{minted}
\note{
\begin{itemize}
\item Why do we use pandas for data analysis? What is matplotlib typically used for?
\item Example: Try importing a library not installed and see the error.
\end{itemize}
}
\end{frame}

% --- Step 2: Declare Data Paths ---
\begin{frame}[fragile]{Step 2: Declare Data Paths}
\begin{minted}[fontsize=\scriptsize]{python}
# Declare paths
tradePath = 'https://github.com/omercadopopular/cgoes/raw/refs/heads/master/trade-undergrad/data/API_NE.TRD.GNFS.ZS_DS2_en_excel_v2_38352.xls'
growthPath = 'https://github.com/omercadopopular/cgoes/raw/refs/heads/master/trade-undergrad/data/API_NY.GDP.PCAP.KD.ZG_DS2_en_excel_v2_122434.xls'
\end{minted}
\note{
\begin{itemize}
\item What would change if you set skiprows=0 when loading?
\item Try loading another Excel file: what errors do you encounter?
\end{itemize}
}
\end{frame}

% --- Step 3: Load and Preview Data ---
\begin{frame}[fragile]{Step 3: Load and Preview Data}
\begin{minted}[fontsize=\scriptsize]{python}
# read the trade and growth data from Excel files
trade = pd.read_excel(tradePath, skiprows=3).drop(columns=['Indicator Name', 'Indicator Code'])
growth = pd.read_excel(growthPath, skiprows=3).drop(columns=['Indicator Name', 'Indicator Code'])
trade
\end{minted}
\vspace{0.5ex}
\textbf{Expected output (from notebook):}
\begin{verbatim}
Country Name Country Code       1960       1961  ...
...
[266 rows x 67 columns]
\end{verbatim}
\note{
\begin{itemize}
\item Identify a row with missing values. What does NaN mean?
\item How many countries are present in the trade dataset?
\end{itemize}
}
\end{frame}

% --- Step 4: Reshape Data to Long Format ---
\begin{frame}[fragile]{Step 4: Reshape Data to Long Format}
\begin{minted}[fontsize=\scriptsize]{python}
# collapse from wide to long format for trade data
trade = trade.melt(id_vars=['Country Name', 'Country Code'],
                   var_name='Year', value_name='Trade')
growth = growth.melt(id_vars=['Country Name', 'Country Code'],
                   var_name='Year', value_name='Growth')
trade
\end{minted}
\vspace{0.5ex}
\textbf{Expected output (from notebook):}
\begin{verbatim}
Country Name Country Code  Year       Trade
...
[17290 rows x 4 columns]
\end{verbatim}
\note{
\begin{itemize}
\item What is the difference between "wide" and "long" data formats?
\item Why do we need to reshape the data before merging?
\end{itemize}
}
\end{frame}

% --- Step 5: Merge DataFrames ---
\begin{frame}[fragile]{Step 5: Merge Trade and Growth Data}
\begin{minted}[fontsize=\scriptsize]{python}
# merge the two dataframes on Country Name, Country Code, and Year
frame = pd.merge(trade, growth, on=['Country Name', 'Country Code', 'Year'])
frame
\end{minted}
\vspace{0.5ex}
\textbf{Expected output (from notebook):}
\begin{verbatim}
Country Name Country Code  Year       Trade     Growth
...
[17290 rows x 5 columns]
\end{verbatim}
\note{
\begin{itemize}
\item What does an "inner join" mean in pandas?
\item Find a country and year where either Trade or Growth is missing.
\end{itemize}
}
\end{frame}

% --- Step 6: Remove Missing Data ---
\begin{frame}[fragile]{Step 6: Remove Rows with Missing Values}
\begin{minted}[fontsize=\scriptsize]{python}
# drop missing values
frame = frame.dropna(subset=['Trade','Growth'])
frame
\end{minted}
\vspace{0.5ex}
\textbf{Expected output (from notebook):}
\begin{verbatim}
Country Name Country Code  Year       Trade     Growth
...
[10871 rows x 5 columns]
\end{verbatim}
\note{
\begin{itemize}
\item What happens if you drop rows with missing data in only one column?
\item How many rows remain after this operation?
\end{itemize}
}
\end{frame}

% --- Step 7: Median Split ---
\begin{frame}[fragile]{Step 7: Categorize Countries by Trade Openness}
\begin{minted}[fontsize=\scriptsize]{python}
# group by year and compute median trade share
median_trade = (frame.groupby(['Year'])['Trade'].transform('median'))

# create a group label (Above / Below median)
frame['group'] = (frame['Trade'] > median_trade).map({True: 'Above Median', False: 'Below Median'})
frame
\end{minted}
\vspace{0.5ex}
\textbf{Expected output (from notebook):}
\begin{verbatim}
Country Name Country Code  Year       Trade     Growth         group
...
[10871 rows x 6 columns]
\end{verbatim}
\note{
\begin{itemize}
\item For a given year, which country has the highest trade share?
\item What are the implications of splitting the dataset at the median?
\end{itemize}
}
\end{frame}

% --- Step 8: Compute Growth Averages ---
\begin{frame}[fragile]{Step 8: Calculate Average Growth by Group}
\begin{minted}[fontsize=\scriptsize]{python}
# take averages across groups
result = frame.groupby(['Year', 'group'])['Growth'].mean().reset_index()
result
\end{minted}
\vspace{0.5ex}
\textbf{Expected output (from notebook):}
\begin{verbatim}
Year         group    Growth
...
[128 rows x 3 columns]
\end{verbatim}
\note{
\begin{itemize}
\item Does "Above Median" group always have higher average growth?
\item Plot one year’s growth rates for both groups.
\end{itemize}
}
\end{frame}

% --- Step 9: Pivot Table ---
\begin{frame}[fragile]{Step 9: Create Pivot Table for Visualization}
\begin{minted}[fontsize=\scriptsize]{python}
# pivot table (as in excel)
pivot = result.pivot(index='Year',
                     columns='group',
                     values='Growth')
pivot.index = pivot.index.astype(int)
pivot['difference'] = pivot['Above Median'] - pivot['Below Median']
pivot
\end{minted}
\vspace{0.5ex}
\textbf{Expected output (from notebook):}
\begin{verbatim}
group  Above Median  Below Median  difference
Year
1961   ...          ...           ...
...
2024   ...          ...           ...
\end{verbatim}
\note{
\begin{itemize}
\item For the year 2020, what is the difference in growth between groups?
\item What does a positive difference mean?
\end{itemize}
}
\end{frame}

% --- Step 10: Visualize Results ---
\begin{frame}[fragile]{Step 10: Visualize Growth Differences}
\begin{minted}[fontsize=\scriptsize]{python}
# chart difference
plt.figure(figsize=(10, 6))
plt.plot(pivot.index, pivot['difference'], marker='o')
plt.axhline(pivot['difference'].mean(), color='red', linewidth=1)  # mean line
plt.axhline(0, color='black', linewidth=1)  # zero line
plt.title('Difference in GDP per Capita Growth by High and Low Trade Country', fontsize=14)
plt.xlabel('Year')
plt.ylabel('GDP per Capita Growth (%)')
plt.xticks(rotation=90)
plt.grid(True, linestyle='--', alpha=0.7)
plt.tight_layout()
plt.show()
\end{minted}
\vspace{0.5ex}
\textbf{Expected output (from notebook):}
\begin{verbatim}
<Figure size 1000x600 with 1 Axes>
\end{verbatim}
\note{
\begin{itemize}
\item What does the red line represent?
\item In which years is the difference most pronounced?
\end{itemize}
}
\end{frame}

% --- Final Reflection Slide ---
\begin{frame}{Reflect and Discuss}
\textbf{Questions:}
\begin{itemize}
    \item What does the analysis suggest about the relationship between trade openness and growth?
    \item What other factors might influence these results?
    \item How would you improve this analysis?
\end{itemize}
\end{frame}

% --- Assignment ---
\begin{frame}{Assignment}
\textbf{Questions:}
\begin{itemize}
    \item Go to \url{https://data.worldbank.org/} or \url{https://fred.stlouisfed.org/}
    \item Explore some data related to trade (it may be cross-sectional or time series)
    \item Download it as a CSV or XLSX, import it to Python
    \item Transform it
    \item Plot a chart
    \item Write a short paragraph summarizing why you think this fact is important.
\end{itemize}
\end{frame}


\end{document}