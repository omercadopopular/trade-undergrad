\documentclass[11pt]{article}
%%%%%%%%%%%%%%%%%%%%%%%%%%%%%%%%%%%%%%%%%%%%%%%%%%%%%%%%%%%%%%%%%%%%%%%%%%%%%%%%%%%%%%%%%%%%%%%%%%%%%%%%%%%%%%%%%%%%%%%%%%%%%%%%%%%%%%%%%%%%%%%%%%%%%%%%%%%%%%%%%%%%%%%%%%%%%%%%%%%%%%%%%%%%%%%%%%%%%%%%%%%%%%%%%%%%%%%%%%%%%%%%%%%%%%%%%%%%%%%%%%%%%%%%%%%%
\usepackage{amssymb}
\usepackage{amsfonts}
\usepackage{amsmath}
\usepackage{geometry}
\usepackage{multirow}
\usepackage{hyperref}
\usepackage{graphicx}
\usepackage{color}


\setcounter{MaxMatrixCols}{10}

\newtheorem{theorem}{Theorem}
\newtheorem{acknowledgement}[theorem]{Acknowledgement}
\newtheorem{algorithm}[theorem]{Algorithm}
\newtheorem{axiom}[theorem]{Axiom}
\newtheorem{case}[theorem]{Case}
\newtheorem{claim}[theorem]{Claim}
\newtheorem{conclusion}[theorem]{Conclusion}
\newtheorem{condition}[theorem]{Condition}
\newtheorem{conjecture}[theorem]{Conjecture}
\newtheorem{corollary}[theorem]{Corollary}
\newtheorem{criterion}[theorem]{Criterion}
\newtheorem{definition}[theorem]{Definition}
\newtheorem{example}[theorem]{Example}
\newtheorem{exercise}[theorem]{Exercise}
\newtheorem{lemma}[theorem]{Lemma}
\newtheorem{notation}[theorem]{Notation}
\newtheorem{problem}[theorem]{Problem}
\newtheorem{proposition}[theorem]{Proposition}
\newtheorem{remark}[theorem]{Remark}
\newtheorem{solution}[theorem]{Solution}
\newtheorem{summary}[theorem]{Summary}
\newenvironment{proof}[1][Proof]{\noindent\textbf{#1.} }{\ \rule{0.5em}{0.5em}}
%\input{tcilatex}
\geometry{left=0.9in,right=0.9in,top=0.7in,bottom=0.9in}
\renewcommand{\baselinestretch}{1.4}
\newcommand{\minitab}[2][l]{\begin{tabular}{#1}#2\end{tabular}}

\begin{document}
	
	\textsc{   }
	\bigskip
	
	\begin{center}
		\Large{\textsc{International Trade Theory and Policy}}
		
		\Large{ECON 2181 - Fall 2025}
		
		\LARGE{Tue-Thu 11:10am-12:25pm, room: xxxx }
		
		\Large{George Washington University}
		
	
		
	%	\large{Office Hours: Tuesdays and Thursdays 11:00am-12:30pm}
		
	\end{center}

%	\begin{center}
%\begin{tabular}{ c }
%	\multicolumn{1}{c}{\textit{Instructor}} \\
%	Giacomo Rondina \\
%	\href{mailto:grondina@ucsd.edu}{grondina@ucsd.edu}\\
%\end{tabular}
	
\medskip
	
	\begin{center}
			Carlos Góes  \\
			\textit{Professorial Lecturer}\\
			cgoes@ucsd.edu\\
	\end{center}	
		\medskip
		
	\bigskip





\begin{center}
\Large{\textsc{Syllabus}} 
\end{center}

\bigskip

\noindent \textit{\noindent Note: This syllabus covers all the important organizational details. I consider the Syllabus my contract with you and I commit to uphold it as written. However,  events outside our control can change the conditions under which the class is taking place. While I do not anticipate doing so, situations may arise that require adjustments of the class rules to ensure that the quality of your learning is preserved, so please consider the information contained in the Syllabus as subject to revision. I commit to keep you promptly informed and included in the decision process.}
\bigskip

\noindent Why is the typical American today 10 times richer than the typical American a century ago? Why are some countries much richer than others? Why do some countries grow fast and catch up with the richest countries while others remain stagnant? Can standards of living keep increasing indefinitely?

\noindent These are arguably among the most important questions studied in macroeconomics. In Econ 110A we will learn how macroeconomists' approach to such questions has evolved over time, what we have come to learn, and what we still only partially understand about the macroeconomy in the long-run.

\noindent In the process, we will also solidify our knowledge of how to measure important macroeconomic objects such as GDP, Inflation, and Unemployment, which you have already learned in Principles of Macroeconomics (Econ 3), and we will learn how to solve and use formal macroeconomic equilibrium models.


\newpage

\bigskip
\noindent\textsc{Learning Objectives}

\smallskip

\noindent The learning objectives in Econ 110A in terms of the core competencies to acquire in the Economics major are:

\smallskip

\noindent \textbf{1. Quantitative Reasoning}: set up and solve optimization models; build macroeconomic models; apply macroeconomic models to understand current issues. 

\medskip

\noindent \textbf{2. Critical Thinking}: explain economic models as deliberate simplifications of reality that economists create to think through complex, nondeterministic behaviors; identify the assumptions and limitations of each model and their potential impacts; select and connect economic models to real
economic conditions; think creatively and combine or synthesize existing economic ideas.

\medskip

\noindent \textbf{3. Written Communication}: write cogent economic arguments. 

\medskip

\noindent The learning objectives for the Economics program can be consulted \href{https://economics.ucsd.edu/undergraduate-program/major-minor-requirements/economics-major-ba.html}{here} and \href{https://dev-academicaffairs.ucsd.edu/_files/ug-ed/asmnt/lo-programs-core-comp/Economics_IEEI_CC.pdf}{here}.

\smallskip

\bigskip
\noindent\textsc{Course Material, Attendance and Practice Problems}

\smallskip

\noindent{The textbook for the class is ``Macroeconomics," 5th Edition, by Charles I. Jones, published by Norton. We will cover chapters 1-8 and 16-19 from Jones' textbook. Lectures will focus on the more difficult material but you will be responsible for all
the material in each chapter. It is strongly recommended to read the required chapters before each lecture. You can find the schedule of lectures below.
During lecture I will present examples/remarks/extensions that do not appear in the textbook but that will be part of the examination material. Following lectures is an individual student responsibility, everything I say during lecture is fair game for the exams, unless noted otherwise. Discussion sessions on material presented during lectures and solutions to the problem sets will be held by our TA's during the session. We may also have a discussion board for the class on Discord, where you can ask questions, submit comments and also help in providing answers on the material covered in class. }


\bigskip
\noindent\textsc{Logistics and Important Dates}

\noindent We meet \textcolor{red}{\textbf{in person}} two times a week on Tuesdays and Thursdays 5:00p-6:20p at PCYNH 122. Video podcast of the lectures will be recorded and recordings will be posted on Canvas. Our TA's will hold discussion sessions \textcolor{red}{\textbf{in person}} every Wednesday 4:00p-4:50p at CSB 001. Office hours times and location will be communicated in Week 1.

\noindent There will be one in-class midterm exam, and one final exam. 

\textcolor{red}{\textbf{All exams will take place in person.}} Please mark your calendar as follows:
\begin{itemize}
\item \textcolor{red}{\textbf{Midterm, May 2\textsuperscript{st}, 5:00 pm to 6:20 pm;}}

\item \textcolor{red}{\textbf{Final, June 13th\textsuperscript{th}, TBD pm;}} 


\end{itemize}

\bigskip
\noindent\textsc{Grading}

\noindent There are 550 points up for grab in this course. Your final grade will be determined according to the following points

\medskip
\begin{tabular}{l r}
Problem Sets & 100 pts   \\
Midterm & 200 pts   \\
Final Exam & 250 pts 
\end{tabular}

\bigskip



\bigskip

\noindent\textsc{Learning Assessment}

\noindent \textit{Exams.} The goal of the exams are to assess your learning of the material in the class, and ultimately your acquisition of the learning objectives specific to Econ 110A. Exams will be a combination of multiple-choice questions, and short-answer questions. Questions will aim to assess your learning at three levels: basic, intermediate, and advanced. To pass the class, you will need to show mastering of the basic questions. To receive an excellent grade in the class (A- or higher) you will need to show mastering of basic and intermediate questions, as well as some of the advanced.

\bigskip

\noindent \textit{Problem sets.} There will be five problem sets that you are expected to submit during this five-week course. The problem sets will be graded to completion and not to accuracy. They are meant to be an incentive for you to go over the material and learn it for yourself. Each of the problem sets is worth 25 points. While you are encouraged to complete every problem set, this means that you will not be penalized if you miss one of the problem sets.

\bigskip

\noindent\textsc{Exams Policies }

\noindent No make-up exams will be given in this class. You must take
the final exam in order to receive a grade in this course. The date is written above,
be sure to mark your calendar. If you miss a midterm
exam without a university accepted excuse, you will receive a grade of zero (0)
for the exam. If you miss a midterm with a university accepted excuse,
the weight of the other exams will be increased accordingly. Excuses for missed exams must be pre-approved by the Instructor (except when this is not possible in an emergency situation). A student who misses an exam due to physical illness will be required to provide documentation from a health
care professional indicating why the student was physically unable to take
the exam. All documentation must be provided to the Instructor within two working days of the end of the emergency. Failure to comply with any of the above in the specified manner will result
in a grade of zero (0) for the exam.

\bigskip
\normalsize

\noindent\textsc{Academic Integrity}

\noindent Students are expected to do their own work, as outlined in the \href{http://ucsd.edu/catalog/front/AcadInt.html}{UCSD policy on Academic Integrity}. All students of UC San Diego are responsible for knowing and adhering to this institution's policy regarding academic integrity. Cheating, plagiarism, fabrication, lying,
bribery, threatening behavior and assistance to acts of academic dishonesty are examples of
behaviors that violate this policy. Ordinarily, a student engaged in any act of academic dishonesty
will receive a failing grade for the course. In addition, all incidents of academic misconduct
shall be reported to the Academic Integrity Office. Depending on its findings, students who are found
to be in violation of the academic integrity policy will be subject to non-academic sanctions,
including but not limited to university probation, suspension, or expulsion. The Academic Integrity Office can be contacted by email at \href{mailto:aio@ucsd.edu}{aio@ucsd.edu} or by telephone at 858-822-2163. Additional information regarding the University Academic Integrity policy is available at \href{https://academicintegrity.ucsd.edu/}{https://academicintegrity.ucsd.edu/}.

\bigskip
\noindent\textsc{University Policies and Resources}

\noindent \textit{Conduct Code.} To foster the best possible working and learning environment, UC San Diego strives to maintain a climate of fairness, cooperation, and professionalism. UC San Diego's \href{https://ucsd.edu/about/principles.html}{Principles of Community} illustrate the expectations of all members of our community. Consistent with such principles, the \href{https://students.ucsd.edu/_files/student-conduct/ucsandiego-student-conduct-code_interim-revisions1-16-18.pdf}{Student Conduct Code} underscores the pride and the values that define UCSD's community,
while providing students with a framework to guide their actions and behaviors. I recommend reviewing the Student Conduct Code to make sure you are familiar with the behavior that is expected from you in class and on campus in general.

\noindent \textit{Disability Accommodations.} Campus policy regarding disabilities
requires that faculty adhere to the recommendations of the Office for Students with Disabilities (OSD). Any student eligible for and
needing academic adjustments or accommodations because of disability should submit to me a letter from OSD describing appropriate adjustments or accommodations and should arrange to meet with me as soon as possible so that arrangements can be made in a timely manner. University policies regarding disabilities are available at
\href{http://disabilities.ucsd.edu/students/}{http://disabilities.ucsd.edu/students/}. Appointments with OSD (phone or in-person) can be made by calling 858.534.4382 or by email at \href{mailto:osd@ucsd.edu}{osd@ucsd.edu}. More information can also be found \href{https://economics.ucsd.edu/undergraduate-program/accommodations-for-students-with-disabilities.html#Step-1:-Request-Academic-Accomm}{here}.

\smallskip
\noindent \textit{Religious Observance.} Campus
policy regarding religious observances requires that faculty make every effort to reasonably and
fairly accommodate all students who, because of religious obligations, have conflicts with
scheduled examinations, assignments or required attendance. See full
details of policies on examinations \href{https://senate.ucsd.edu/operating-procedures/educational-policies/courses/epc-policies-on-courses/policy-exams-including-midterms-final-exams-and-religious-accommodations-for-exams/}{here.} If you
have scheduling conflict covered by this policy, please let me know as soon as possible so that we can reschedule the
relevant assignment/examination.

\smallskip
\noindent \textit{Harassment Policy.} The {\href{http://ucsd.edu/catalog/front/shpp.html}{University Policy on Discrimination and
Harassment} applies to all students, staff and faculty. Any student, staff member or faculty member who believes (s)he has been the subject of
discrimination or harassment based on race, color, national origin, sex, pregnancy, age, disability, creed, religion, sexual orientation, gender identity, gender expression, veteran status, political affiliation, or political philosophy, should
contact the Office for the Prevention of Harassment and Discrimination (OPHD) at (858) 534-8298, \href{mailto:ophd@ucsd.edu}{ophd@ucsd.edu}, or \href{http://ophd.ucsd.edu/report-bias/index.html}{reportbias.ucsd.edu}.

\smallskip
\begin{sloppypar}
\noindent \textit{Data Privacy.} The University adheres to the standards for student privacy rights and requirements as stipulated
in the Federal Rights and Privacy Act (FERPA) of 1974, see \href{http://ucsd.edu/catalog/front/ferpa.html}{http://ucsd.edu/catalog/front/ferpa.html}.
\end{sloppypar}

\smallskip
\begin{sloppypar}
\noindent \textit{Counseling.} Managing the many challenges of being a college student can be very stressful. Always remember that talking to somebody that is professionally trained to help is just an email, or short walk, away: \href{https://caps.ucsd.edu/}{https://caps.ucsd.edu/}.
\end{sloppypar}


\pagebreak
\begin{center} \textsc{} \end{center}


\begin{center}
	\begin{tabular}{|c|c|l|c|}
		
		\hline \textbf{Date} & \textbf{Lecture} & \textbf{Topic} & \textbf{Book Chapter} \\ \hline
			
		Tue, Aug 26 & Lecture 1 & Intro \& The Neoclassical Model of Consumption &  16.2 \\ \hline
		Thu, Aug 28 & Lecture 2  & Measuring the Economy & 2.1-2.4 \\ \hline
		Tue, Sep 02 & Lecture 3 & Long-Run Growth: Facts & 3.1-3.7 \\ \hline
	    Thu, Sep 04 &Lecture 4 & A Model of Production & 3.4-3.7 \\ \hline	
		Tue, Sep 09 & Lecture 5 & A Model of Production & 4.1-4.5 \\ \hline
		Thu, Sep 11 & Lecture 6 & The Solow Growth Model: Analysis & 5.1-5.4 \\ \hline
		Tue, Sep 16 &  Lecture 7 & The Solow Growth Model: Experiments & 5.5-5.7 \\ \hline
		
		Thu, Sep 18 & Lecture 8 & How do Firms Make Investment Decisions? & 17.2 \\ \hline

		Tue, Sep 23 & Lecture 9 & Review Ahead of Midterm & \textemdash \\ \hline

  	Thu, Sep 25 &  Lecture 10 & x & \textemdash  \\ \hline

        Tue, Sep 30 & Lecture 11 & The Romer Model & 6.3-6.5 \\ \hline

        Thu, Oct 2 & Lecture 12 & Solow + Romer & 6.4, 6.9 \\ \hline

        Tue, Oct 7 & Lecture 13 & Labor Measurement &  7.3-7.5 \\ \hline

        Thu, Oct 9 & Lecture 14  &  \textbf{ } &   \\ \hline

        Tue, Oct 14 & Lecture 15 & Human Capital: The Lucas Model and Growth Accounting &7.6-7.7  \\ \hline
        Thu, Oct 16 & Lecture 16 & Money and Prices &8.1-8.3  \\ \hline
		
        Tue,  Oct 21 & Lecture 17 & Inflation, The Marginal Propensity to Consume & 8.4-8.6, 16.3-16.4 \\ \hline

        Thu, Oct 23 & Lecture 18 & Trade (time permitting)   & 19   \\ \hline

        Tue, Oct 28 & Lecture 19 & Intertemporal trade (time permitting) & Lecture 20  \\  \hline

        Thu, Oct 30 & Lecture 21 & Trade (time permitting)   & 19   \\ \hline

        Tue, Nov 4 & Lecture 22 & Intertemporal trade (time permitting) &  \\  \hline

        Thu, Nov 6 & Lecture 23 & Trade (time permitting)   & 19   \\ \hline

        Tue, Nov 11 & Lecture 24 & Intertemporal trade (time permitting) & 19  \\  \hline

        Thu, Nov 13 & Lecture 25 & Trade (time permitting)   & 19   \\ \hline

        Tue, Nov 18 & Lecture 26 & Intertemporal trade (time permitting) & 19  \\  \hline

        Thu, Nov 20 & Lecture 27 & Trade (time permitting)   & 19   \\ \hline

        Tue, Nov 25 &  & \textbf{Thanksgiving Break} &   \\  \hline

        Thu, Nov 27 &   & \textbf{Thanksgiving Break}    & 19   \\ \hline

        Tue, Dec 2 &  Lecture 27 & &   \\  \hline

        Thu, Dec 4 &  Lecture 28 &    & 19   \\ \hline



	 \hline
		
		%\textbf{Thu December $13^{th}$} & \textbf{Final Exam, Room TBA} & \textbf{All} \\ \hline
		
	\end{tabular}
\end{center}

%* No in-person class, pre-recorded lecture will be posted on Canvas.



\end{document}
