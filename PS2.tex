\documentclass[11pt,letterpaper]{exam}

\newtheorem{proposition}{Proposition}
\newcommand{\blue}[1]{\textcolor{blue}{#1}}
\newcommand{\white}[1]{\textcolor{white}{#1}}

\usepackage{tikz}
\usetikzlibrary{shapes.geometric}
\usepackage{pgfplots}
\usetikzlibrary{patterns, pgfplots.fillbetween}
\usepackage{graphicx}
\usepackage{verbatim}
\usepackage{subfigure}
\usetikzlibrary{positioning}
\usetikzlibrary{snakes}
\usetikzlibrary{calc}
\usetikzlibrary{arrows}
\usetikzlibrary{decorations.markings}
\usetikzlibrary{shapes.misc}
\usetikzlibrary{matrix,shapes,arrows,fit,tikzmark}
\usepackage{amsmath}
\usepackage{mathpazo}
\usepackage{hyperref}
\usepackage{lipsum}
\usepackage{multimedia}
\usepackage{multirow}
\usepackage{dcolumn}
\usepackage{bbm}
\usepackage{comment}
\usepackage{booktabs}
\usepackage{tabularx}
\usepackage{adjustbox}
\usepackage{multicol}
\usepackage{mathtools}
\usepackage[table,xcdraw]{xcolor}
\usepackage[top=0.5in, headsep=0pt]{geometry}
\usepackage{lastpage}
\cfoot{Page \thepage \hspace{1pt} of \pageref{LastPage}}

\newcommand{\figpath}{figs/}

\usepackage{titlesec}
\titleformat*{\section}{\large\bfseries}

\begin{document}
\begin{center}
\Large{\textsc{International Trade Theory and Policy}}\\[4pt]
\Large{ECON 2181 \;--\; Fall 2025}\\[6pt]
\large Carlos Góes, Ph.D. \\
\textit{Professorial Lecturer}\\
\href{mailto:c.bezerradegoes@gwu.edu}{c.bezerradegoes@gwu.edu} \\
\end{center}

\bigskip

\begin{center}
\Large{\textsc{Problem Set 2}}
\end{center}


\begin{questions}

\question Consider a world with 2 countries $i \in \{ H, F\}$ and multiple goods indexed by $g \in [0,1]$. In country $i$, there are $L_i$ units of labor (worker-hours) available, which we call the labor endowment. An inherent technological characteristic of country $i$ are \textbf{unit labor requirements}, i.e. to produce one unit of good $g$ firms in country $i$ use $a_{i,g}$ units of labor. In country $i$, firms producing good $g$ maximize profits under perfect competition. Assume there are no shipping costs across countries.

\begin{parts}
\part Write down the profit maximization function of a firm producing good $g$ in country $i$. Assume firms take prices $P_g$ as given. Give an economic interpretation to each term in the function the firm maximizes.

\part Recall that in perfect competition profits for each good are equal to zero. Given that information, express the price of good $P_g$ as a function of cost of production. Give an economic interpretation for the result.

\part If good $g$ is produced at home, its cost of production is $w_H a_{H,g}$. If good $g$ is produced in the foreign country, its cost of production is $w_F a_{F,g}$. Using the relationship between prices and cost of production you derived above, explain where goods will be produced by comparing the cost of production in each country.

\part In the simple Ricardian model, we conclude that country $F$ has a comparative advantage if $\frac{a_{F,g}}{a_{H,g}} \le  \frac{a_{F,g'}}{a_{H,g'}}$. Rewrite the inequality you derive above in terms of $\frac{a_{F,g}}{a_{H,g}}$.

\part Let $A_g \equiv \frac{a_{F,g}}{a_{H,g}}$. Suppose we order goods $[0,1]$ according to their comparative advantage, such that $0$ is the good in which home is the most productive (in relative terms); $1$ is the good in which home is the least productive (in relative terms); and each good in between changes accordingly. Is $A_g$ a decreasing or an increasing function? Explain your reasoning and draw $A_g$ on the $y$-axis as a function of $g$ on the $x$-axis.

\part Home spends share $G$ of its income on goods produced at home and share $1-G$ on goods produced abroad. By perfect competition, both countries spend a share $G$ of their income on goods produced at $H$. Since labor is the only factor of production, then:
\[
w_H L_H = G \times w_H L_H + G \times w_F L_F.
\]
Solve for $w_H/w_F$ as a function of $G$ and parameters.

\part Draw this function as an increasing function of $G$. Draw in the chart which share of goods is produced at home, which share of goods is produced abroad, and what is the equilibrium relative wage.

\part What happens if the foreign population increases to $L_F'>L_F$?

\end{parts}

\question Let $F(K,L)$ be a Cobb–Douglas production function and $0 < \beta < 1$.

\begin{parts}
    \part Write down the Cobb–Douglas production function.

    \part Prove mathematically that the Cobb–Douglas production function is constant returns to scale and explain in words what that means.

    \part Write down the firm's profit maximization problem in the production model for a firm whose technology is Cobb–Douglas. Specify choice variables and discuss the intuition.

    \part Define the marginal product of labor (MPL) and, using the production function above, show that MPL is decreasing in labor.

    \part Draw the chart of $Y$ as a function of $L$, holding $K$ fixed. Explain the shape of the chart and state the economic concept it illustrates.

    \part Derive the optimality conditions for choices $K$ and $L$ in the maximization problem of the firm. Interpret economically what they mean.

    \part Suppose that labor supply is $L^S = \bar{L} = 1$ and capital supply is $K^S = \bar{K} = 1$. Using (i) the optimality conditions, (ii) the production function, (iii) $P=1$, (iv) $\bar{Z}=1$, $\beta=1/2$, and (v) market clearing, solve for equilibrium $(w,r,Y)$.
\end{parts}
\end{questions}

\end{document}
