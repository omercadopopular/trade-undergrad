\documentclass[11pt]{article}
%%%%%%%%%%%%%%%%%%%%%%%%%%%%%%%%%%%%%%%%%%%%%%%%%%%%%%%%%%%%%%%%%%%%%%%%%%%%%%%%%%%%%%%%%%%%%%%%%%%%%%%%%%%%%%%%%%%%%%%%%%%%%%%%%%%%%%%%%%%%%%%%%%%%%%%%%%%%%%%%%%%%%%%%%%%%%%%%%%%%%%%%%%%%%%%%%%%%%%%%%%%%%%%%%%%%%%%%%%%%%%%%%%%%%%%%%%%%%%%%%%%%%%%%%%%%
\usepackage{amssymb}
\usepackage{amsfonts}
\usepackage{amsmath}
\usepackage{geometry}
\usepackage{multirow}
\usepackage{hyperref}
\usepackage{graphicx}
\usepackage{color}
\usepackage[T1]{fontenc}

\setcounter{MaxMatrixCols}{10}

\newtheorem{theorem}{Theorem}
\newtheorem{acknowledgement}[theorem]{Acknowledgement}
\newtheorem{algorithm}[theorem]{Algorithm}
\newtheorem{axiom}[theorem]{Axiom}
\newtheorem{case}[theorem]{Case}
\newtheorem{claim}[theorem]{Claim}
\newtheorem{conclusion}[theorem]{Conclusion}
\newtheorem{condition}[theorem]{Condition}
\newtheorem{conjecture}[theorem]{Conjecture}
\newtheorem{corollary}[theorem]{Corollary}
\newtheorem{criterion}[theorem]{Criterion}
\newtheorem{definition}[theorem]{Definition}
\newtheorem{example}[theorem]{Example}
\newtheorem{exercise}[theorem]{Exercise}
\newtheorem{lemma}[theorem]{Lemma}
\newtheorem{notation}[theorem]{Notation}
\newtheorem{problem}[theorem]{Problem}
\newtheorem{proposition}[theorem]{Proposition}
\newtheorem{remark}[theorem]{Remark}
\newtheorem{solution}[theorem]{Solution}
\newtheorem{summary}[theorem]{Summary}
\newenvironment{proof}[1][Proof]{\noindent\textbf{#1.} }{\ \rule{0.5em}{0.5em}}

\geometry{left=0.9in,right=0.9in,top=0.7in,bottom=0.9in}
\renewcommand{\baselinestretch}{1.2}
\newcommand{\minitab}[2][l]{\begin{tabular}{#1}#2\end{tabular}}

\begin{document}

\begin{center}
\Large{\textsc{International Trade Theory and Policy}}\\[4pt]
\Large{ECON 2181 \;--\; Fall 2025}\\[6pt]
\LARGE{Tue--Thu \, 11:10am--12:25pm, Monroe Hall (Room MON B32)}\\[4pt]
\Large{George Washington University}
\end{center}

\medskip

\begin{center}
\large Carlos Góes, Ph.D. \\
\textit{Professorial Lecturer}\\
\href{mailto:c.bezerradegoes@gwu.edu}{c.bezerradegoes@gwu.edu} \\
\end{center}

\bigskip

\begin{center}
\Large{\textsc{Syllabus}}
\end{center}

\bigskip

\noindent \textit{Note: This syllabus summarizes course organization and policies. I will uphold it as written. However, unforeseen events may require adjustments to safeguard learning quality. If updates are necessary, I will communicate changes promptly and transparently.}

\bigskip

% ------------------------------------------------------------------
\noindent\textsc{Course Description and Materials}

\smallskip
This course studies the foundations of international trade and policy: classical and neoclassical theories of comparative advantage, the specific-factors and Heckscher–Ohlin models, firm heterogeneity and intra-industry trade, trade policies under perfect and imperfect competition, and contemporary topics linking theory to real-world applications. 

\medskip
\textbf{Textbook (required):} Paul Krugman,  Maurice Obstfeld, Marc Melitz. International Trade: Theory and Policy. 11th edition.

\bigskip

% ------------------------------------------------------------------
\noindent\textsc{Learning Objectives}

\smallskip
By the end of the semester, you should be able to:
\begin{enumerate}
    \item \textbf{Comprehend and explain} the modern trade theories and their assumptions;
    \item \textbf{Illustrate} each theory using standard analytical tools (e.g., PPFs, GE diagrams, gravity) and data;
    \item \textbf{Solve} for the equilibrium of different models, using pen-and-paper as well as a computer;
    \item \textbf{Interpret policy implications} and distributional effects of trade and trade policy.
\end{enumerate}

\bigskip

% ------------------------------------------------------------------
\noindent\textsc{Prerequisites}
\smallskip

ECON 1011 and ECON 1012 (or equivalents).

\bigskip

% ------------------------------------------------------------------
\noindent\textsc{Logistics and Attendance}

\smallskip
We meet \textbf{in person} on Tuesdays and Thursdays, 11:10am--12:25pm, in Monroe Hall (room  MON B32). Lectures may be recorded subject to room availability and university policy; recordings, when available, will be posted on Blackboard.

\bigskip

% ------------------------------------------------------------------
\noindent\textsc{Assessments and Grading}

\smallskip
Course grades are based on two midterms, one final, code assignments, and two mini-projects:

\medskip
\begin{tabular}{l r}
Code assignments & 15\% \\
Mini-Project 1 & 5\% \\
Mini-Project 2 & 5\% \\
Midterm & 30\% \\
Final Exam (cumulative) & 45\%
\end{tabular}

\medskip
\noindent Details, rubrics, and deadlines will be posted online. Limited extra credit may be awarded for \textit{constructive, regular class participation}.

\bigskip

% ------------------------------------------------------------------
\noindent\textsc{Mini-Projects (Concise Briefs)}

\smallskip
\textbf{Mini-Project 1 (Research Snapshot).} Choose a recent (past 5 years) research-based policy piece (e.g., VoxEU, VoxDev) related to international trade. Submit \textbf{3 slides} summarizing the question, method/intuition, key findings, and policy implications. Upload slides to the shared class folder (Blackboard link provided). 

\smallskip
\textbf{Mini-Project 2 (Country Profile for a Client).} Prepare a \textbf{5-slide} memo advising a hypothetical client on exporting to or investing in a country of your choice. Discuss market size/demand, supply/competition, trade/industrial policies, risk factors, and recommendations. Include at least two charts/tables.

\smallskip
\textit{Tentative deadlines:} Mini Project 1 - at any time before the mid-term; 

\bigskip

% ------------------------------------------------------------------
\noindent\textsc{Code assignments and Data Labs}

\smallskip
We will have a series of Data Labs throughout the semester. The idea is to introduce students to the Python programming language and its uses for data science and economics. You are expected to download and install the \href{https://www.anaconda.com/download}{Anaconda Navigator} environment on your machine (there are Windows, Linux, and MacOS versions). We will have five data labs. After three of them, you will be required to submit a coding assignment. Each will be worth $5\%$ of your grade. 

\smallskip
\textit{Tentative deadlines:} one week after each data lab that requires an assignment.

\bigskip

% ------------------------------------------------------------------
\noindent\textsc{Problem sets}

\smallskip
I will assign a subset of the textbooks exercises as practice problem sets. They will not be graded but will serve as practice for your midterm and final. I will use some exercises that show up in your problem sets in the exams.


\bigskip


% ------------------------------------------------------------------
\noindent\textsc{Independent Learning \& Use of Generative AI}

\smallskip
\textbf{Independent learning.} For a 3-credit course, plan for \textbf{a minimum of 5 hours/week} of independent study on top of class time. Actual needs vary by background and goals. See GW’s credit hour policy in the University Bulletin.

\smallskip
\textbf{Generative AI policy.} Unless I explicitly prohibit the use of generative AI, you can use it for this course. AI can be quite useful as a tutor or helper in studying. I use it daily, particularly for coding or deep research. Note the following: (i) you are responsible for everything you submit; (ii) ``ChatGPT'' is not a source -- you need to go to original sources and verify the claims; (iii) you should ground your claims on reputable sources (peer reviewed research; reports from the IMF, World Bank, WTO, OECD, etc); (iv) you need to type/write the answers yourself -- in other words, you can't copy and paste images or texts out of ChatGPT into the answers

\bigskip

% ------------------------------------------------------------------
\noindent\textsc{Course Policies}

\smallskip
\textbf{Exams.} No make-up exams are offered. You can choose to skip the midterm and increase the weight of your final exam. Late homework is not accepted without a qualifying emergency. Final exam scheduling follows the official GW schedule.

\smallskip
\textbf{Collaboration.} You are encouraged to discuss problem sets conceptually with peers, but the work you submit must be your own. All submitted work must reflect your independent understanding.

\bigskip

% ------------------------------------------------------------------
\noindent\textsc{University Policies and Student Resources}

\smallskip
\textbf{Academic Integrity Code.} Academic dishonesty includes cheating, misrepresentation of work, and fabrication of information. Students should conform to \href{https://students.gwu.edu/code-academic-integrity}{GW’s Code of Academic Integrity}. Students are responsible for becoming familiar with the different forms that plagiarism can take.  Ignorance does not exempt students from being penalized for plagiarism.  Plagiarism is a serious matter both inside academia and outside it.

\smallskip
\textbf{Plagiarism and How to Avoid It}: A good overview of plagiarism and how to avoid it can be found at http://widstudents.wordpress.com/tag/plagiarism/  It is worth reading through the entire web page, including the section titled "Plagiarism Tales at GW."   On the proper use of quotations, see http://writingcenter.unc.edu/resources/handouts-demos/citation/quotations

\smallskip
\textbf{Writing Help at GW}: GW provides students with help in writing, through the University Writing Center, http://www.gwu.edu/~gwriter/ and the EAP Writing Support Program https://eap.columbian.gwu.edu/.  Students have benefited from these services and they should be encouraged to take advantage of them.

\smallskip
\textbf{Observance of Religious Holidays.} Inform me during Week 1 of any religious observances that conflict with course requirements; reasonable accommodations will be made per university policy.

\smallskip
\textbf{Disability Support Services (DSS).} Students needing accommodations should contact DSS (Rome Hall, 801 22nd St NW, Suite 102; 202-994-8250) to establish eligibility and coordinate accommodations.

\smallskip
\textbf{Counseling \& Psychological Services.} The Colonial Health Center provides counseling and psychological services (202-994-5300).

\smallskip
\textbf{Safety and Security.} In an emergency: call GWPD 202-994-6111 or 911. See the Emergency Response Handbook and “Stay Informed” resources on the GW safety site.

\bigskip

% ------------------------------------------------------------------
\noindent\textsc{Tentative Schedule (subject to updates on Blackboard)}

\medskip
\begin{center}
\begin{tabular}{|c|c|l|c|}
\hline \textbf{Date} & \textbf{Type} & \textbf{Topic} & \textbf{Readings / Notes} \\ \hline
			
		Tue, Aug 26 & Lecture 0 & Intro, presentations \& logistics & Syllabus \\ \hline
		Thu, Aug 28 & Lecture 1  & Why do we trade? & Ch. 3; Handout 1 \\ \hline
		Tue, Sep 02 & Data Lab 1 & Intro to python & Notebook 1 \\ \hline
	    Thu, Sep 04 &Lecture 2 & Intro to Classical Ricardian Trade & Ch. 3; Handout 1 \\ \hline	
		Tue, Sep 09 & Lecture 3 & Classical Ricardian Trade in General Equilibrium (i) & Ch. 3; Handout 1 \\ \hline
		Thu, Sep 11 & Lecture 4 & Classical Ricardian Trade in General Equilibrium (ii) & Ch. 3; Handout 1 \\ \hline
		Tue, Sep 16 & Lecture 5 & Ricardian Trade with Multiple Goods & Handout 2 \\ \hline
		
		Thu, Sep 18 & Lecture 6 & Empirics of Comparative Advantage &  \textemdash  \\ \hline

		Tue, Sep 23 & Data Lab 2 & Intro to pandas; trade with data & Ch. 1; Notebook 2 \\ \hline

  	Thu, Sep 25 &  Lecture 7 & Production in the Short and the Long Run & \textemdash \\ \hline

        Tue, Sep 30 &  Lecture 8 & Specific Factors Model (i) & Ch. 4; Handout 3  \\ \hline

        Thu, Oct 2 &  Lecture 9 & Specific Factors Model (ii) & Ch. 4; Handout 3 \\ \hline

        Tue, Oct 7 & Lecture 10 & The Political Economy of Trade Policy &  Ch. 10 \\ \hline

        Thu, Oct 9 &Data Lab 3 & Solving the SFM in the computer & Notebook 2 \\ \hline

        Tue, Oct 14 & Review & Midterm Review  &   \\ \hline
        Thu, Oct 16 &  & \textcolor{red}{\textbf{Midterm}} &   \\ \hline
		
        Tue,  Oct 21 &Lecture 11 & The Heckscher-Ohlin Model (i)  & Ch. 5; Handout 4  \\ \hline

        Thu, Oct 23 &  Lecture 12 & The Heckscher-Ohlin Model (ii) & Ch. 5; Handout 4  \\ \hline 

        Tue, Oct 28 &  Data Lab 4 & Solving the HO model in the computer & Notebook 3 \\ \hline

        Thu, Oct 30 & Lecture 13 & The Standard Trade Model, Gravity, and Welfare   &  Ch. 6   \\ \hline
        
        Tue, Nov 4 & Data Lab 5 & The Gravity Regression & Notebook 4 \\  \hline

        Thu, Nov 6 & Lecture 14 & Economies of Scale & Ch. 7   \\ \hline

        Tue, Nov 11 & Lecture 15 & The Krugman Model (i) & Ch. 8; Handout 5 \\  \hline

        Thu, Nov 13 &Lecture 16 & The Krugman Model (ii)   & Ch. 8; Handout 5   \\ \hline

        Tue, Nov 18 &Lecture 17 & The Melitz Model (i) & Handout 6  \\  \hline

        Thu, Nov 20 & Lecture 18 & The Melitz Model (ii)   &  Handout 6   \\ \hline

        Tue, Nov 25 &  & \textbf{Thanksgiving Break} &   \\  \hline

        Thu, Nov 27 &   & \textbf{Thanksgiving Break}    &    \\ \hline

        Tue, Dec 2 & Lecture 19 & The Instruments of Trade Policy & Ch. 9 \\  \hline

        Thu, Dec 4 &  Lecture 20 & TBD &    \\ \hline
\end{tabular}
\end{center}

\smallskip
\noindent \textit{Final exam:} Per official GW exam schedule (time/place TBA on Blackboard).

\end{document}
